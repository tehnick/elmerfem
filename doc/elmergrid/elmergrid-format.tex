

\chapter{ElmerGrid format}

The native mesh definition format of the program is the ElmerGrid
format that has the suffix \texttt{.grd}. 
If the user gives a name for the input file that doesn't exist
a simple geometry file is automatically created by the program.
This may then be edited by any text editor to suite the needs of the particular case.

The ElmerGrid parser is not very bright. It knows only
a handful of keywords. If the default values are 
used the corresponding commands may be neglected. 
The commands usually assume a format where the 
the keyword is followed by its arguments. 
If the arguments are on the same line then a equality sign 
``='' should be used in between. 
The parser is not case sensitive and thus the keywords may be written with 
capital or small letters, or any combination of them.


\section{ElmerGrid keywords}

\sifbegin
\sifitem{Version}{210903}
Gives the version of the current input file. This should not be changed unless
the format is changed. The parser should be downward compatible also with the older
format of the input file but the new format is preferred.
\sifitem{Coordinate System}{String}
The choices are Cartesian 1D, Cartesian 2D, Cartesian 3D, Polar and 
Axisymmetric. Axisymmetric and Cartesian 2D have currently
no difference.
A cartesian 3D grid is the default.
%
\sifitem{Subcell Divisions}{Integer[DIM]} 
The mesh is composed of subcells that are 
topologically squares. 
This keywords should be followed by the 
number of subcells in the main directions,
$n_1$, $n_2$ and $n_3$.
Thus the number of 
arguments may be 1, 2 or 3 depending on the dimension of the geometry 
definition. 
\sifitem{Subcell Limits i}{Real[]}
The size of the subcells are defined by $n_i+1$ boundaries
in each coordinate direction. These values define the physical 
dimensions of the mesh before possible manipulation. 
This keyword exists for all active coordinate directions.  
%
\sifitem{Subcell Sizes i}{Real[]}
An alternative way to give the subcell sizes is to give the size of 
each subcell in each coordinate direction. 
The subcell limits are then the cumulative sum of these sizes.
%
\sifitem{Subcell Origin i}{Real[]}
When the subcells are defined by the sizes the user may also define the origin 
in each direction. The origin may be a real number or also some of the following 
strings, \texttt{Left}, \texttt{Center}, \texttt{Right}.
%
\sifitem{Material Structure}{Integer[][]}
The subcells form a $n_1 \times n_2$ matrix in 2D. 
The material structure of this plane may therefore be given by a matrix 
consisting of integers relating to different materials.
These numbers may then later be referred
when creating the mesh and boundary conditions.
In 1D the structure is simply a vector.
In 3D the material structure refers to the structure of the 
$xy$-plane before extrusion of rotation takes place. 
%
\sifitem{Materials Interval}{Integer[2]} 
Elements are created only for those materials that 
lie in the interval defined by the two integers.
This is on easy way to define different meshes for different physical
phenomena.
In 3D the interval refers to the materials that should be present
in the extrusion. 
%
\sifitem{Extruded Sructure}{}
The keyword may be used to define the layered 
structure of the geometry. Otherwise the extruded structure is just the same
as the material structure in the plane. 
The definition must include $n_3$ lines with three integers on each. 
The first two integers define the materials interval that is extruded and
the third one the new material index of the extruded layer.
Materials outside the interval are not extruded and therefore 
materials may selectively be dropped out during extrusion process. 
If the new material has value zero then the corresponding material layer in the 
extruded structure remains unaltered.
This feature may, for example, be useful in MEM-applications. 

When the 2D geometry is extruded the rectangular elements 404, 408 and 409 result to 
elementtypes are 808, 820 and 827 (using Elmer notation). 
Also linear triangles may be extruded and they result to 
linear prisms of type 606.
%
\sifitem{Geometry Mappings}{}
The grid is by default rectangular since all the subcells are also
rectangular. Non-rectangular shapes may be introduced my mapping
the straight lines separating the subcells to lines represented by 
analytical functions. 
The mapping may be performed in the $xy$-plane or out of the plane.
The number of the mappings is calculated automatically and therefore 
the field should be ended by \texttt{End}.

If the mode of the mapping is more than 50 then 
the mapping is performed out of plane while mappings 
in the plane is the default.
Mapping out of the plane makes the 2D plates to become 3D shells.

This keyword activates mappings as in the case of mappings in plane 
but now the mapping is performed 
out of the plane. 
Thus the 2D geometry becomes 3D geometry.
This way some simple 3D shells may be created. 
If in addition, the coordinate system is set to be polar this 
third coordinate is treated as radius $r$ and the second
coordinate as angle $\phi$. 
%
\sifitem{Polar Radius}{Real}
If the coordinate system is chosen to be polar then 
the values defining the are $(z,\theta)$.
When saving the mesh these are mapped into cartesian coordinates.
There may be on offset in the radius when using the polar coordinates.
This value is then added to the value of the $r$-coordinate before making the
coordinate transformation. This would result into simple pipes, for example.
In addition there is a possibility to have the radius change as a function
of position using special mappings.
%
\sifitem{Revolve Radius}{Real}
There are three keywords that may be used to define how a 2D mesh is revolved
around the $y$-axis
to create 3D cylindrical meshes.
Defining any of the keywords enforces revolation.
Simple rotation would result to singular elements at the 
origin. Therefore a square block is set at the origin.
By default this square has the same $x$-coordinate as the
first subcell in $x$-direction.
In rotation the material structure remains unaltered. 
%
\sifitem{Revolve Blocks}{Integer}
The keyword has a parameter 1, 2, 3 or 4 depending on 
how many degrees of rotation is desired. The values 
correspond to rotation of 90, 180, 270 or 360 degrees.
If there is less than 4
blocks the program finds the sides that form the 
symmetry axis and creates additional boundaries from them.   
%
\sifitem{Revolve Improve}{Real}  
This keyword controls the quality of the rotated mesh.
The rotation creates elements that are not rectangular as
the whole angle of 360 degrees needs to be divided between 3
elements. The default value for this keyword is one 
which means that the inner block is a square.
The value zero results to the angle being equally divided to three 
120 degree angles by the three elements.
%
\sifitem{Revolve Curve Direct}{Real}
The rotate structure may have a curve. This defines the 
direct part of the curve.
\sifitem{Revolve Curve Angle}{Real}
This defines the angle of the curve.
\sifitem{Revolve Curve Radius}{Real}
This defines the radius of the curve.
%
\sifitem{Boundary Definitions}{}
There may be several boundaries. Each is defined by a line
of four parameters. Note that this field should be finished by 
\texttt{End}. 

Each boundary requires a \texttt{type} that refers 
to the number of the boundary condition.
It may then later be referred to when setting 
up the boundary conditions.
The boundaries are created in such a way that
the material \texttt{int} should have elements
created. The \texttt{out} may be an empty subcell.
If the value of the flag \texttt{double} is 2 or more 
secondary nodes on the
boundary will be created. This enables the use of
discontinuous boundary conditions.

It is also possible to create boundary conditions on a 
given side of a material. The number $-1$ refers to \textbf{down},
$-2$ to \textbf{right}, $-3$ to \textbf{top} and $-4$ to \textbf{left}. 
This means that the boundary is created on the given side 
of the material that is specified by the other index.

In order to make the creation of the boundary condition
more robust there are some special numbers that may be used.
$-11$ refers to \textbf{smaller}, $-9$ to \textbf{bigger}, 
and $-10$ to \textbf{anything}. For example, if the 
\texttt{intmat} is 5 and the \texttt{outmat} is $-9$, then
the boundary is created for all materials that have a larger
material number than 5 and have a common boundary with 
material 5. This makes it possible to create several 
material boundaries with one definition.
%
\sifitem{Numbering}{Vertical/Horizontal}
The keyword enforces the numbering of nodes and elements.
The numbering affects the 
bandwidth of the resulting matrix equation.
The numbering should rather be performed by the shorter axis.
%
\sifitem{Surface Elements}{Integer}
This keyword may be used to roughly determine the number
of elements on the surface. This unique feature makes it possible to
change the mesh density by altering just one figure in the
mesh definition file.
Using the constraints on the relative mesh densities 
the program then tries to match this given number.  
%
\sifitem{Volume Elements}{Integer}
In 3D it is not as easy to obtain the desired 
number of elements. In this case the desired number of surface elements
if altered iteratively until also the number of volume elements is close to the
desired one.
%
\sifitem{Volume Nodes}{Integer}
Instead of trying to meet the desired number of elements one may also 
try to optimize the number of nodes using this keyword.
%
\sifitem{Element Degree}{Integer}
The mesh generator may produce linear, quadratic and 
cubic elements. 
The elements are initially always topologically rectangles and
may have 4, 5, 8, 9, 12 or 16 nodes. 
%
\sifitem{Element Midpoints}{Logical}
For higher order elements the user may choose whether to
create nodes inside the elements. For example a second order rectangle 
may have 8 or 9 nodes. The default is \texttt{False}.
%
\sifitem{Triangles}{Logical}
ElmerGrid makes originally structured quadrilateral meshes.
These may be divided into triangles 
using the shorter of the diagonals.
The resulting mesh is still structured and
often inferior in computations. For some purposes triangular meshes 
may however be favored. The default is \texttt{False}.

The quadrilateral elements may have 4, 5, 8, 9 or 16 nodes. The 
elements with 4, 9  and 16 nodes are divided into two and the element with
5 nodes into four, correspondingly. The 8-node element cannot be divided.

If the mesh is also extruded the triangles become prisms. In this case
only linear triangles are applicable.


\sifitem{Elementtype}{Integer}
An alternative way to define the elementtype is to define the 
element type as in Elmer software.
The allowed types are 303, 306, 310, 404, 405, 408, 409,
412 and 416. Then the three flags above are tuned accordingly.
%
\sifitem{Coordinate Ratios}{Real[DIM-1]} 
By default the relative density of the coordinate 
directions is the same. This keyword affects the 
mesh density ratio between different directions.
Thus the dimension is one less than the space dimensions.
%
\sifitem{Element Ratios i}{Real[]}
Defines the ratio of the first and last element in the particular
subcell in direction \texttt{i}. The default value is 1.
Does not compare the mesh density with other subcells.
The ratio must be given for each coordinate direction 
separately. Value close to unity leads to linear meshing.
Negative values activate symmetric density profile within the subcell. 
%
\sifitem{Element Densities i}{Real[]}
This keyword refers 
to densities compared to other subcells in direction \texttt{i}.
The default value is 1. A larger value means local 
mesh refinement on that particular subcell.
%
\sifitem{Minimum Element Divisions}{Integer[DIM]} 
This keyword defines the minimum number of elements 
within each subcell for each coordinate direction.
The default is 1 1 2, this is also the set of minimum values
that may be used.
%
\sifitem{Element Divisions i}{Integer} 
This keywords sets the number of elements in direction \texttt{i} in the subcells
absolutely. Therefore it overrides the desired number of elements
and the relative element densities. By default the relative 
mesh density definition is used. 
%
\sifitem{Start New Mesh}{}
The format includes the possibility to generate multiple
meshes using one input file. This keyword invokes the 
start of a new mesh. All properties are inherited from the 
previous mesh and therefore the new mesh may be 
defined only by few changes.
\sifend



\section{Command file keywords}

Sometimes the necessary commands in the using the inline parameters
should create an exhaustive and noninformative list. Therefore most of the
inline parameters may also be given in an command file.
Here the description only deals only with the command file
syntax. For other information on these commands a more accurate description will 
be found in the previous chapter where the inline arguments where explained.

Note that these commands may also be put to the end of the ElmerGrid mesh 
definition file.

\sifbegin
\sifitem{Input File}{String}
The name of the input file.
The default name the same as the name of the command file (if any) 
attached with an appropriate suffix. 
%
\sifitem{Output File}{String}
The name of the input file.
The default name is the same as the name of the input file
attached with an appropriate suffix. 
%
\sifitem{Input Mode}{String/Integer}
The name of the Input mode or the corresponding integer. 
The choices are ElmerGrid(1), ElmerSolver(2), ElmerPost(3), 
Ansys(4), Ideas(5), Abaqus(6), Fidap(7), Easymesh(8), 
FemLab(9), Fieldview(10) and Triangle(11).
%
\sifitem{Output Mode}{String/Integer}
The name of the Input mode or the corresponding integer. 
The choices are ElmerGrid(1), ElmerSolver(2), ElmerPost(3).
%
\sifitem{Decimals}{Integer}
The number of decimals in output.
%
\sifitem{Triangles}{Logical}
Divide rectangles into triangles.
%
\sifitem{Merge Nodes}{Real}
Merges nodes close to one another, $|\vec{r}-\vec{r}_0| < \varepsilon$.
%
\sifitem{Unite}{Logical}
Unite different meshes, default is \texttt{False}.
%
\sifitem{Order Nodes}{Real[DIM]}
Orders the nodes using $\vec{c}\cdot\vec{r}$ as the measure.
%
\sifitem{Scale}{Real[DIM]}
Scale the mesh coordinates. 
%
\sifitem{Translate}{Real[DIM]}
Translate the mesh with the given vector.
%
\sifitem{Rotate}{Real[DIM]}
Rotate the mesh around the main axis with the given degrees.
%
\sifitem{Clone}{Integer[DIM]}
Make a multiple copies of the mesh.
%
\sifitem{Clone Size}{Real[DIM]}
The size of the mesh to be cloned.
%
\sifitem{Mirror}{Real[DIM]}
Mirror the mesh around the main axis.
%
\sifitem{Polar Radius}{Real}
Gives the radius of the 2D mesh in polar coordinates 
and also enforces the new coordinate system.
%
\sifitem{Reduce Degree}{Integer[2]}
Reduces the element degree in the given interval.
%
\sifitem{Increase Degree}{Logical}
Increase the degree. Default is \texttt{False}.
%
\sifitem{Power Elements}{}
This keyword activates the p-elements. 
There may be several p-element definition. Is 
has a separate line and defines with three integers the interval
on which the p-element is applied and the power of the element.
The field should be ended with \texttt{End}.
%
\sifitem{Partition}{Integer[DIM]}
Activates the simple mesh partitioning.
Gives the number of divisions in each coordinate 
direction.
%
\sifitem{Partition Order}{Real[DIM]}
The ordering used in making the simple partitioning.
%
\sifitem{Metis}{Integer}
Activates the Metis partitioning.
%
\sifitem{Periodic}{Integer}
Enforces periodic boundary conditions with the partitioning algorithms.
%
\sifitem{Material Boundary}{}
Define boundaries between materials. 
Each boundary is an a separate line
and the field should be ended with \texttt{End}
%
\sifitem{Boundary Boundary}{}
Define boundaries between boundaries.
Each boundary is an a separate line
and the field should be ended with \texttt{End}
%
\sifitem{Renumber Material}{}
Renumbers the materials on the given interval to be of desired type.
Each renumbering is an a separate line
and the field should be ended with \texttt{End}
%
\sifitem{Renumber Boundary}{Integer[3]}
Renumbers the boundaries on the given interval to be of desired type.
Each renumbering is an a separate line
and the field should be ended with \texttt{End}
%
\sifitem{Isoparametric}{Logical}
Sets the intermediate nodes in elements to be in between 
the corner nodes.
%
\sifitem{No Boundary}{Logical}
Disables the saving of boundary information.
Default is \texttt{False}.
%
\sifitem{Extruded Divisions}{Integer} 
A two-dimensional mesh may be extruded to multiple layers. This
keyword gives the number of subcell in the extruded direction $n_3$.
\sifitem{Extruded Limits}{Real[$n_3+1$]}
The dimensions of the extruded layers.
\sifitem{Extruded Elements}{Integer[$n_3$]}
The number of elements in each extruded layer.
\sifitem{Extruded Ratios}{Real[$n_3$]}
Defines the ratio of the first and last element in the particular
subcell in the extruded direction.
\sifitem{Extruded Structure}{}
Functionality as in the \texttt{ElmerGrid} geometry format.
%
\sifend



\section{Mapping modes}
The use of different mapping options is quite an 
unintuitive matter. Unfortunately it is the only way to 
create non-rectangular shapes in ElmerGrid. 
The mesh mapping is activated by the keyword \texttt{Geometry Mappings}.
There are several mapping modes that define the type of mapping.

Most of the mappings operate on the lines separating different subcells.
Positive mode refers to mapping of horizontal lines, 
and negative mode to mapping of vertical lines.
The limits of the mapping define the distance at which the 
mapping of the line settles down. If the distance is large compared to 
the size of geometry then the same mapping is performed to the 
whole geometry. 

Several mappings may be applied at the same time. The 
coordinates are then mapped as a linear combination of the 
individual mappings. Mappings may also be performed 
consecutively up to five levels. This is obtained by adding 
the figure 10 to the mode. Mappings of the same 
level are done at the same time while mappings of different levels
are done at on increasing order of modulo 10s.

In order to perform mappings out of plane a value
of 50 may be added in the mapping mode.

The parameters used in the mapping vary from case to case.
Note that if a vertical line is mapped the coordinates $x$ and $y$
are switched.
\begin{description}
\item[1: piecewise linear] \mbox{} \\
Parameters are $(x_0,dy_0)$, $(x_1,dy_1)$, $(x_2,dy_2)$,~$\ldots$
Between the points linear interpolation is used. 
\item[2: piecewise power law] \mbox{} \\
Parameters are
$(x_0,dy_0)$, $b_1$, $(x_1,dy_1)$, $b_2$, $(x_2,dy_2)$,~$\ldots$
Between the points the values for $dy$ are interpolated from
$a_i (x-x_i)^{b_i}$.
\item[3: piecewise circle arc] \mbox{} \\
Parameters are
$(x_0,dy_0)$, $r_1$, $(x_1,dy_1)$, $r_2$, $(x_2,dy_2)$,~$\ldots$
Between the points the values for $dy$ are calculated assuming that
there is a circle with radius $r_i$ drawn through the lines.
If radius is larger than zero the origin of the circle is 
above the mapped line and vice versa.
\item[4: line to circle] \mbox{} \\
Parameters are $x_0$, $r_0$, $x_1$, $r_1$,~$\ldots$
Different from other models also the other coordinate is 
transformed. In order to get a circle with radius $r_i$ 
as a result there must initially be a square with 
side length $2r_i$. The line to be mapped is originally 
one of the sides of this squares.
\item[5: line to sinus] \mbox{} \\
Parameters are $x_0$, $x_1$, $c$, $dy_1$,
where $c$ is the number of waves between $x_0$ and $x_1$
and $dy_1$ is the height of the wave.
\item[6: line to power series] \mbox{} \\	
Parameters are $x_0$, $x_1$, $a_0$,\ldots,$a_i$
and the displacement is then $\sum_i a_i u^i$,
where $u$ is $x$ normalized between 0 and 1. 
\item[7: line to power series variation] \mbox{} \\	
Parameters are $x_0$, $x_1$, $a_0$,\ldots,$a_n$
and the displacement is then $\sum_i a_i p_i(u)$,
where $u$ is $x$ normalized between 0 and 1.
The polynom $p_0=1$, $p_1=u$, $p_2=u(u-1)$, 
$p_3=u(u-1/2)(u-1)$, $p_4=u(u-1/3)(u-2/3)(u-1)$,~$\ldots$
\item[8: piecewise constant angle] \mbox{} \\
Parameters are $(x_0,\phi_0)$, $(x_1,\phi_1)$, $(x_2,\phi_2)$,~$\ldots$
This is used to create 3D shells only. 
The angle of the shell is $\phi_i$ if $x \in [x_i,x_{i+1}]$. 
May be used to create polygonal shells.

\end{description}

