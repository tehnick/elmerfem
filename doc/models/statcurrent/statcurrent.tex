\chapter{Static Current Conduction}\label{chap:statcurr}

\modinfo{Module name}{\Idx{StatCurrentSolve}}
\modinfo{Module subroutines}{\Idx{StatCurrentSolver}}

\begin{versiona}
\modinfo{Module authors}{Leila Puska, Antti Pursula, Peter R�back}
\modinfo{Document authors}{Antti Pursula}
\modinfo{Document edited}{August 2nd 2002}


\section{Introduction}

The macroscopic electromagnetic theory is governed by the Maxwell's
equations. This module solves the electrostatic potential in
conducting medium allowing volume currents and electric power loss
(Joule heating) to be derived.


\section{Theory}

In quasi-static approximation, when $\varepsilon \mu c^2 << 1$, the
first and fourth Maxwell equation can be written as follows
\begin{eqnarray}
\nabla\cdot \Vec{D} & = & \rho \label{max1} \\
%\nabla\cdot \Vec{B} & = & 0 \\
%\nabla\times \Vec{E} & = & -\frac{\partial\Vec{B}}{\partial t} \\
\nabla\times \Vec{H} & = & \Vec{J} + \frac{\partial\Vec{D}}{\partial t}
\label{max4}
\end{eqnarray}

The continuity equation for electric charges is
easily derived from the above Maxwell Eqs.~\ref{max1} and~\ref{max4}
\begin{equation}
\frac{\partial \rho}{\partial t} + \nabla\cdot \Vec{J} = 0
\label{elcontinuity}
\end{equation}
The \Idx{Ohm's law} for conducting material gives
the relationship between \Idx{current density} and 
electric field,
\begin{equation}
\Vec{J} = \sigma \Vec{E}
\label{ohm}
\end{equation}
In steady-state case the electric field may be expressed with a help of
an electric scalar potential $\phi$, 
\begin{equation}
\Vec{E} = -\nabla \phi . 
\label{elpot}
\end{equation}

Starting from the continuity equation~\ref{elcontinuity} and using
Eqs.~\ref{ohm} and~\ref{elpot} we get
\begin{equation}
\nabla\cdot\sigma\nabla\phi = \frac{\partial\rho}{\partial t}.
\end{equation}
This Poisson equation is used to solve the electric potential. The
source term is often zero but in some cases it might be necessary.

The volume current density is now calculated by
\begin{equation}
\vec{J} = -\sigma\nabla\phi,
\end{equation}
and electric power loss density which is turned into heat by
\begin{equation}
h = \nabla\phi\cdot\sigma\nabla\phi.
\end{equation}
The latter is often called the \Idx{Joule heating}. The total heating power
is found by integrating the above equation over the conducting
volume.

\subsection{Boundary Conditions}

For electric potential either Dirichlet or Neumann boundary condition
can be used. The Dirichlet boundary condition gives the value of the
potential on specified boundaries. The Neumann boundary condition is
used to give a current $J_b$ on specified boundaries
\begin{equation}
J_b = \sigma\nabla\phi\cdot\vec{n}.
\end{equation}


\section{Note on output control}

The user can control which derived quantities ({\em i.e.} volume
current and Joule heating) are calculated and additionally specify if
he/she wants to output also the electric conductivity. The latter is
useful when the conductivity depends for example on temperature. This
feature is available only for isotropic (scalar) conductivities.

There are also available two choises of visualization types for the
derived quantities. The node values can be calculated by taking the
average of the derived values on neighbouring elements (constant
weights). This results often in visually good images. The other
possible choice is to weight the average with the size of the
elements, which is more accurate and should be used when some other
variable depends on these derived values. The latter choice is also
the default.


\section{Keywords}
\end{versiona}

\sifbegin
\sifitemnt{Solver}{solver id}
\sifbegin
\sifitemnt{Equation}{String Stat Current Solver}
\sifitem{Variable}{String Potential}
This may be of any name as far as it is used consistently also elsewhere.
\sifitem{Variable DOFs}{Integer 1}
Degrees of freedom for the potential.
\sifitem{Procedure}{File "StatCurrentSolve"\ "StatCurrentSolver"}
Following are listed three keywords with default values
for output control.
\sifitemnt{Calculate Volume Current}{Logical [True]}
\sifitemnt{Calculate Joule Heating}{Logical [True]}
\sifitem{Calculate Electric Conductivity}{Logical [False]}
This keyword can be used to write the electric conductivity into
results. Might be useful when conductivity depends on some other
quantity such as temperature.
\sifitem{Constant Weights}{Logical [True]}
Used to turn constant weighting on for the results.
\sifend

\sifitemnt{Material}{mat id}
\sifbegin
\sifitemnt{Electric Conductivity}{Real}
\sifitem{Density}{Real}
The density of the material is needed only when also the
heat equation is solved, and Joule heating is used as body force,
since heat equation uses heating power per mass as body force, whereas
Joule heating is calculated normally as power per volume.
\sifend

\sifitemnt{Body Force}{bodyforce id}
\sifbegin
\sifitem{Current Source}{Real}
Possibility for a body force, not used often though.
\sifend

\sifitemnt{Boundary Condition}{bc id}
\sifbegin
\sifitem{Potential}{Real}
Dirichlet BC for the potential.
\sifitem{Current Density BC}{Logical}
Must be set to {\tt True} if Neumann BC is used.
\sifitem{Current Density}{Real}
Neumann boundary condition for the current.
\sifend
\sifend

\begin{versiona}
\bibliography{elmerbib}
\bibliographystyle{plain}
\end{versiona}

