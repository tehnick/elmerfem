\chapter{Vorticity Computation}

\modinfo{Module name}{\Idx{VorticitySolver}}
\modinfo{Module subroutines}{VorticitySolver}
\modinfo{Module authors}{Peter R�back}
\modinfo{Document authors}{Peter R�back}
\modinfo{Document edited}{14.2.2008}


\section{Introduction}

This module is used to calculate the vorticity of vector fields.
Vorticity may be of interest mainly in the postprocessing of 
flow fields or electromagnetic fields.

\section{Theory}

The vorticity $\Vec{w}$ of a vector field $\Vec{v}$ is obtained 
simply from the curl of the field,
\begin{equation}
  \Vec{w} = \nabla \times \Vec{v} .
  \label{eq:vorticity}
\end{equation}
Component-wise the equations for the vorticity read 
\begin{eqnarray}
  w_x & = & \Der{v_z}{y} - \Der{v_y}{z} \\
  w_y & = & \Der{v_x}{z} - \Der{v_z}{x} \\
  w_z & = & \Der{v_y}{x} - \Der{v_x}{y} .
  \label{eq:vorticity2}
\end{eqnarray}
Thus, all three components exist only in 3D while in 2D and axisymmetric
cases only the $z$-component is present

The most consist ant method for computing the vorticity 
in conjunction with the finite elements is to solve the 
equations~(\ref{eq:vorticity2}) using the Galerkin method.
The resulting matrix is diagonally dominated and 
may be computed easily with iterative methods even with poor
preconditioners. In 3D
the vorticity computation may be done component-wise so that each 
component $w_i$, where $i=1,2,3$, is solved separately.
This saves some memory and may also save 
in the overall time consumption.
If only one component is
desired in the 3D computations then the $z$-component is computed.

\section{Keywords}
\sifbegin
\sifitemnt{Solver}{solver id}
\sifbegin
\sifitemnt{Equation}{String Vorticity Solver}
\sifitemnt{Procedure}{File "VorticitySolver"\ "VorticitySolver"}
\sifitem{Vorticity Variable}{String "Velocity"}
This gives the name of the vector variable used to compute the 
vorticity. By default the variable is \texttt{Velocity}.
\sifitem{Constant Bulk Matrix}{Logical}
This keyword may be used to activate the saving of the stiffness matrix if the same solver
is called repeatedly. The stiffness matrix depends only on geometric information and is hence the same
if the geometry is unaltered.
\sifend
%
The following keywords are not usually needed as they are set by the
initialization procedure of the solver.
\sifbegin
\sifitem{Variable}{String "-nooutput tempvar"}
The variable is usually 
only used to allocate the corresponding matrix. Therefore output is not 
required unless the solver is only used to compute one component.
\sifitem{Vorticity Result Variable}{String f1}
This string gives the name of the variable that is known to be at 
disposal for saving the results. This variable is allocated with 
the following keyword.
\sifitem{Exported Variable 1}{String "fl[Vorticity:3]"}
This command is used to allocate space for the result and at the 
same time the components may be renamed to be later identified as a vector
in ElmerPost. If only one component of the flux is computed this keyword
is obsolete.
\sifend
The solver is easily solved even without preconditioning. Fox example, the
following linear system control may be applied. 
\sifbegin
\sifitemnt{Linear System Solver}{"Iterative"}
\sifitemnt{Linear System Iterative Method}{"cg"}
\sifitemnt{Linear System Preconditioning}{None}
\sifitemnt{Linear System Max Iterations}{500}
\sifitemnt{Linear System Convergence Tolerance}{1.0e-10}
\sifend
\sifend
