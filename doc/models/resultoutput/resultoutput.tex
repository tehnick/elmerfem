\Chapter{Result Output for Other Postprocessors}

\noindent
\modinfo{Module name}{\Idx{ResultOutputSolve}}
\modinfo{Module subroutines}{ResultOutputSolver}
\begin{versiona}
\modinfo{Module authors}{Erik Edelmann, Mikko Lyly, Peter R�back}
\modinfo{Document authors}{Peter R�back}
\modinfo{Document created}{11.12.2006}
\modinfo{Document edited}{3.8.2007}

\section{Introduction}

This subroutine is intended for saving data in other than the native format 
of Elmer -- ElmerPost. The reason for using another postprocessing tool might
be that some feature is missing in ElmerPost, or that the user is more aquinted 
with some other visuzalition software. Currently supported formats include
\Idx{GiD}, \Idx{Gmsh}, \Idx{VTK} legacy, XML coded VTK file bearing the suffix 
\Idx{VTU} and \Idx{Open DX}.



\section{Keywords}
\end{versiona}

\sifbegin
\sifitemnt{Solver}{solver id}
\sifbegin
\sifitem{Equation}{String "ResultOutput"}
The name of the equation. This is actually not much needed 
since there are no degrees of freedom associated with this solver.
\sifitem{Procedure}{File "ResultOutputSolve"\ "ResultOutputSolver"}
The name of the file and subroutine. 
%
\sifitem{Output File Name}{File}
Specifies the name of the output file. 
%
\sifitem{Output Format}{String}
This keyword the output format of choice. 
The choices are \texttt{gid, gsmh, vtk, vtu}, and \texttt{dx}. 
%
\sifitemnt{Gid Format}{Logical}
\sifitemnt{Gmsh Format}{Logical}
\sifitemnt{Vtk Format}{Logical}
\sifitemnt{Vtu Format}{Logical}
\sifitem{Dx Format}{Logical}
The user may also use the above logical keywords to set which of the formats is saved. 
This has more flexibility in that there may be several formats that are saved
simultaneously where the \texttt{Output Format} keyword may only be used to activate one 
solution type. 
\sifend
The following keywords related only to the GiD, Vtu and Gsmh formats. In the other formats all available 
degrees of freedom are saved. 
\sifbegin
\sifitem{Scalar Field i}{String}
The scalar fields to be saved, for example \texttt{Pressure}. 
Note that the fields must be numbered continously starting from one.
\sifitem{Vector Field i}{String}
The vector fields to be saved, for example \texttt{Velocity}
\sifitem{Tensor Field i}{String}
The tensor fields to be saved. The rank of tensor fields should be 
3 in 2D and 6 in 3D. 
\sifend
\sifend

