\chapter{Navier-Stokes Equation}

\modinfo{Module name}{included in solver}
\modinfo{Module subroutines}{\Idx{FlowSolve}}
\modinfo{Module authors}{Juha Ruokolainen}
\modinfo{Document authors}{Juha Ruokolainen, Peter R�back}
\modinfo{Document edited}{10.8.2004}

\section{Introduction}
In solid and liquid materials heat transfer and viscous fluid flow are 
governed by heat and \Idx{Navier-Stokes equation}s, which can be derived from the 
basic principles of conservation of mass, momentum and energy. Fluid can be
either \Idx{Newtonian} or \Idx{non-Newtonian}. In the latter case the consideration in
Elmer is limited to purely viscous behaviour with the power-law model.

In the following we present the governing equations of fluid flow, heat
transfer and stresses in elastic material applied in  Elmer. Also the most
usual boundary conditions applied in computations are described.

\section{Theory}

%\subsection{The Governing Equations}

The momentum and continuity equations can be written as
\begin{eqnarray}
\rho \left( 
\frac{\partial\vec u}{\partial t} + (\vec u \cdot \nabla)\vec u
\right) - \nabla\cdot \overline{\overline \sigma} = \rho\vec f,
\label{momentum}
\end{eqnarray}
and
\begin{eqnarray}
\left( \frac{\partial\rho}{\partial t} + (\vec u\cdot \nabla)\rho
\right)  + \rho(\nabla\cdot\vec u) = 0,
\label{continuity}
\end{eqnarray}
where $\overline{\overline \sigma}$ is the stress tensor. For Newtonian
fluids
\begin{eqnarray}
\overline{\overline\sigma} = 2\mu \overline{\overline\varepsilon}
-\frac{2}{3} \mu (\nabla\cdot\vec u)\overline{\overline I} - p 
\overline{\overline I},
\label{newtonian}
\end{eqnarray}
where $\mu$ is the viscosity, $p$ is the pressure, $\overline{\overline I}$
the unit tensor and $\overline{\overline \varepsilon}$ the linearized strain
rate tensor, i.e.
\begin{eqnarray}
\varepsilon_{ij} = \frac{1}{2}\left( \frac{\partial u_i}{\partial x_j} +
\frac{\partial u_j}{\partial x_i}
\right). \label{linstrain}
\end{eqnarray}
The density of an ideal gas depends on the pressure and temperature through
the equation of state
\begin{eqnarray}
\rho = \frac{p}{RT}\label{stateequation},
\end{eqnarray}
where $R$ is the gas constant:
\begin{eqnarray}
R = \frac{\gamma - 1}{\gamma}c_p.
\end{eqnarray}
The specific heat ratio $\gamma$ is defined as
\begin{eqnarray}
\gamma = \frac{c_p}{c_v},\label{specificheatratio}
\end{eqnarray}
where $c_p$ and $c_v$ are the heat capacities in constant pressure
and volume, respectively. The value of $\gamma$ depends solely on
the internal molecular properties of the gas.

An imcompressibe flow is characterized by the condition
$\rho$=constant, from which it follows that
\begin{eqnarray}
\nabla\cdot \vec u = 0. \label{incompressible}
\end{eqnarray}
Enforcing the constraint (\ref{incompressible}) in (\ref{momentum}),
(\ref{continuity}) and (\ref{newtonian}), the equations reduce to
the Navier-Stokes equations
\begin{eqnarray}
\rho \left( \frac{\partial\vec u}{\partial t}
+ (\vec u \cdot\nabla) \vec u \right)
 -\nabla\cdot(2\mu \overline{\overline\varepsilon})+
\nabla p & = &\rho\vec f, \label{NS-equation} \\
\nabla\cdot\vec u &=& 0.
\label{Navier-Stokes_equations}
\end{eqnarray}
Compressible flows are modelled by the equations
(\ref{momentum})-(\ref{specificheatratio}). Then,
it is possible to replace the state equation
(\ref{stateequation}) by
\begin{equation}
\rho = \frac{1}{c^2}p, \label{statespeed}
\end{equation}
where $c=c(p,T,\dots)$ is the speed of sound. The equation
(\ref{statespeed}) can be used with liquid materials as well.

Most commonly the term $\rho\vec f$ represents a force due to gravity, 
in which case the vector $\vec f$ is the gravitational acceleration.
It can also represent, for instance, the \Idx{Lorentz force} when magnetohydrodynamic 
effects are present. 

For isothermal flows the equations
(\ref{NS-equation}) and (\ref{Navier-Stokes_equations}) desrcibe the system
in full. For thermal flows also the heat equation needs to be solved.

For thermal incompressible fluid flows we  assume that the Boussinesq approximation is 
valid. This means that the density of the fluid is constant except in the body
force term where the density depends linearly on temperature through the 
equation
\begin{equation}
\rho = \rho_0 (1-\beta (T-T_0 )),
\end{equation}
where $\beta$ is the volume expansion coefficient and the subscript 0 refers 
to a reference state. Assuming that the gravitational acceleration $\vec g$
is the only external force, then the force $\rho_0 \vec g (1-\beta (T-T_0 ))$
is caused in the fluid by temperature variations. This phenomenon is called
\Idx{Grashof convection} or \Idx{natural convection}.


One can choose between transient and steady state analysis.
In transient analysis one has to set, besides boundary conditions, also 
initial values for the unknown variables. 

\subsection{Boundary Conditions}

For the Navier-Stokes equation one can apply boundary conditions 
for velocity components or the tangential or
normal stresses may be defined.

In 2D or axisymmetric cases the \Idx{Dirichlet boundary condition} for velocity 
component $u_i$ is simply
\begin{equation}
u_i = u_i^b. 
\end{equation}
A value $u_i^b$ can be constant or a function of time, position or other 
variables. In cylindrical cases the Dirichlet boundary condition for angular 
velocity $u^\theta$ is
\begin{equation}
u^\theta =\omega,
\end{equation}
where $\omega$ is the rotation rate.

In axisymmetric geometries one has to set $u_r =0$ and 
$\partial u_z/\partial r =0$  on the symmetry axis.

If there is no flow across the surface, then
\begin{equation}
\vec u\cdot\vec n =0
\end{equation}
where $\vec n$ is the outward unit normal to the boundary.

Surface stresses can be divided into normal and tangential stresses. Normal 
stress is usually written in the form
\begin{equation}
\sigma_n ={{\gamma}\over R} -p_a 
\label{normal_stress}
\end{equation}
where $\gamma$ is the surface tension coefficient, $R$ the mean curvature and 
$p_a$ the atmospheric (or external) pressure. Tangential stress has the form
\begin{equation}
\vec\sigma_\tau = \nabla_s \gamma,
\label{tangential_stress}
\end{equation}
where $\nabla_s$ is the surface gradient operator.

The coefficient $\gamma$ is a thermophysical property depending on the
temperature. Temperature differences on the surface influence the 
transport of momentum and heat near the surface. This phenomenon is called 
Marangoni convection or thermocapillary convection. The temperature
dependence of the surface tension coefficient can be approximated by a linear
relation:
\begin{equation}
\gamma =\gamma_0 (1-\vartheta (T-T_0 )),
\label{surface_tension_coefficient}
\end{equation}
where $\vartheta$ is the temperature coefficient of the surface tension and
the subscript $0$ refers to a reference state. If a Boussinesq
\index{Boussinesq approximation}
hypothesis is made, i.e., the surface tension coefficient is constant except in
(\ref{tangential_stress}) due to (\ref{surface_tension_coefficient}), the boundary
condition for tangential stress becomes
\begin{equation}
\vec\sigma_\tau =-\vartheta\gamma_0\nabla_s \gamma.
\end{equation}
In equation (\ref{normal_stress}) it holds then that $\gamma =\gamma_0$.
The linear temperature dependence of the surface tension coefficient is 
naturally only one way to present the dependence. In fact, the 
coefficient $\gamma$ can be any user defined function in Elmer.
One may also give the force vector on a boundary directly as in
\begin{equation}
\overline{\overline\sigma}\cdot \vec n = \vec g.
\label{boundary_stress}
\end{equation}

\subsection{Linearization}

As is well known, the convective transport term of the Navier-Stokes equations
and the heat equation is a source of both physical and numerical instability.
The numerical instability must be compensated somehow in order
to solve the equations on a computer. For this reason the so called stabilized
finite element method (\cite{franca92},\cite{franca92b}) is used in Elmer to discretize
these equations.  
%The equations in the form implemented in Elmer and
%the discretization are described in more detail in Appendix
%\ref{chapter-solver-discretation}. 
%In Appendix \ref{chapter-free-surfaces} the free surface problem is presented.

The convection term of the Navier-Stokes equations is nonlinear and has to be
linearized for computer solution. There are two linearizations 
of the convection term in Elmer:
\begin{equation}
(\vec u\cdot\nabla)\vec u \approx
(\vec {\cal U}\cdot\nabla)\vec {\cal U}
\end{equation}
and
\begin{equation}
(\vec u\cdot\nabla)\vec u \approx
(\vec u\cdot\nabla)\vec{\cal U} +
(\vec{\cal U}\cdot\nabla)\vec u -
(\vec {\cal U}\cdot\nabla)\vec {\cal U},
\end{equation}
where $\vec{\cal U}$ is the velocity vector from the previous iteration.
The first of the methods is called \Idx{Picard iteration} 
or the method of the fixed point, while the latter
is called \Idx{Newton iteration}. The convergence rate of the Picard iteration is of first order, and the
convergence might at times be very slow. The convergence rate of the Newton method is of second order,
but to succesfully use this method, a good initial
guess for velocity and pressure fields is required. The solution to this problem is to
first take a couple of Picard iterations, and switch to Newton iteration after
the convergence has begun.


\subsection{Non-newtonian Material Models}

There are several non-newtonian material models. All are functions 
of the strainrate $\dot{\gamma}$. The simple power law model has a problematic 
behavior at low shear rates. The more complicated models provide a 
smooth transition from low to high shearrates.

{\bf Power law}
\begin{equation}
  \eta = \begin{cases} 
\eta_0 \dot{\gamma}^{n-1} 		& \text{if $\dot{\gamma} > \dot{\gamma}_0$}, \\
         \eta_0  \dot{\gamma}_0^{n-1}   & \text{if $\dot{\gamma} \le \dot{\gamma}_0$}.
  \end{cases}
\end{equation}
where $\eta_\infty$ is constant, $\dot{\gamma}_0$ is the critical shear rate,
and $n$ is the viscosity exponent. 

{\bf Carreau-Yasuda}
\begin{equation}
  \eta = \eta_\infty + \Delta \eta \left ( 1+(c\dot{\gamma})^y \right )^{\frac{n-1}{y}},
\end{equation}
where $\eta_\infty$ is the high shearrate viscosity $\dot{\gamma}\rightarrow\infty$
provided that $n<1$. For shearrates approaching zero the viscosity is 
$\eta_0=\eta_\infty+\Delta \eta$. $\Delta \eta$ is thus the maximum
viscosity difference between low and high shearrate.
This model recovers the plain Carreau model when the Yasuda exponent $y=2$.

The model can be made temperature dependent. One choice is to multiply $\Delta \eta$ and $c$
by factor $\exp(d(1/T-1/T_0))$, where $d$ and $T_0$ are model parameters.

{\bf Cross}
\begin{equation}
  \eta = \eta_\infty + \frac{\Delta \eta}{1+c\dot{\gamma}^n},
\end{equation}
where again $\eta_\infty$ is the high shearrate viscosity.

{\bf Powell-Eyring}
\begin{equation}
  \eta = \eta_\infty + \Delta \eta \frac{\text{asinh} (c \dot{\gamma} )}{c\dot{\gamma}}.
\end{equation}


\subsection{Flow in Porous Media}

A simple \Idx{porous media} model is provided in the Navier-Stokes solver.
It utilizes the \Idx{Darcy's law} that states that the flow resistance is
proportinal to the velocity and thus the modified momentum equation reads
\begin{eqnarray}
\rho \left( 
\frac{\partial\vec u}{\partial t} + (\vec u \cdot \nabla)\vec u
\right) - \nabla\cdot \overline{\overline \sigma} + r \vec{u}= \rho\vec f,
\label{momentum}
\end{eqnarray}
where $r$ is the porous resistivity which may also be an orthotropic tensor.
Usually the given parameter is permeability which
is the inverse of the resistivity as defined here.
No other features of the porous media flow is taken into consideration.
Note that for large value of $r$ only the bubble stabilization 
is found to work. 

\subsection{Coupling to Electric Fields}

In \Idx{electrokinetics} the fluid may have charges that are coupled
to external electric fields. This results to an external force that is 
of the form 
\begin{equation}
  \vec{f}_e = -\rho_e \nabla \phi ,
\end{equation}
where $\rho_e$ is the charge density and $\phi$ is the external electric field.
The charge density may also be a variable. More specifically this force may be 
used to couple the Navier-Stokes equation to the \Idx{Poisson-Boltzmann equation} 
describing the charge distribution in electric doubly layers. Also
other types of forces that are proportional to the gradient of the field
may be considered.


\subsection{Coupling to Magnetic Fields}

If the fluid has free charges it may couple with an magnetic field.
The magnetic field induced force term for the
flow momentum equations is defined as
\begin{equation}
\vec{f}_m = \vec{J}\times\vec{B},
\end{equation}
Here $\vec B$ and $\vec E$ are the magnetic and electric
fields, respectively. The current density $\vec J$ is defined as
\begin{equation}
\vec J = \sigma(\vec E + \vec u\times \vec B).
\end{equation}



\section{Keywords} 

\sifbegin
\sifitemnt{Constants}{}
\sifbegin
\sifitem{Gravity}{Size 4 Real [x y z abs]}
The above statement gives a real vector whose length is four. In this case the
first three components give the direction vector of the gravity and the fourth
component gives its intensity.
\sifend

\sifitem{Solver}{solver id} 
Note that all the keywords related to linear solver (starting with {\tt Linear System}) may be used in this solver as well.
They are defined elsewhere. 

\sifbegin
\sifitem{Equation}{String [Navier-Stokes]} 
The name of the equation.
\sifitem{Nonlinear System Convergence Tolerance}{Real} this keyword gives a criterion to
terminate the nonlinear iteration after the relative change of the norm of the field variable
between two consecutive iterations is small enough
$$
 ||u_i-u_{i-1}|| < \epsilon ||u_i||,
$$
where $\epsilon$ is the value given with this keyword.
\sifitem{Nonlinear System Max Iterations}{Integer} 
The maxmimum number of nonlinear iterations the
solver is allowed to do.
\sifitem{Nonlinear System Newton After Iterations}{Integer} 
Change the nonlinear solver type to
Newton iteration after a number of Picard iterations have been performed. If a given
convergence tolerance between two iterations is met before the iteration count is met,
it will switch the iteration type instead.
\sifitem{Nonlinear System Newton After Tolerance}{Real} 
Change the nonlinear solver type to
Newton iteration, if the relative change of the norm of the field variable meets a
tolerance criterion:
$$
 ||u_i-u_{i-1}|| < \epsilon ||u_i||,
$$
where $\epsilon$ is the value given with this keyword.
\sifitem{Nonlinear System Relaxation Factor}{Real} Giving this keyword triggers the use
of  relaxation in the nonlinear equation solver.
Using a factor below unity is sometimes required to achive convergence of the nonlinear system.
A factor above unity might speed up the convergence. Relaxed variable is defined as follows:
$$
 u^{'}_i = \lambda u_i + (1-\lambda) u_{i-1},
$$
where $\lambda$ is the factor given with this keyword. The default value for the relaxation factor
is unity.
\sifitem{Steady State Convergence Tolerance}{Real}
With this keyword a equation specific steady state or coupled system
convergence tolerance is given.
All the active equation solvers must meet their own tolerances before the 
whole system is deemed converged.
The tolerance criterion is:
$$
 ||u_i-u_{i-1}|| < \epsilon ||u_i||,
$$
where $\epsilon$ is the value given with this keyword.
\sifitem{Stabilize}{Logical} 
If this flag is set true the solver will use stabilized finite element method
when solving the Navier-Stokes equations.
Usually stabilization of the equations must be done in order to succesfully solve the equations.
If solving for the compressible Navier-Stokes equations, a bubble function formulation
is used instead of the stabilized formulation regardless of the setting of this keyword.
Also for the incompressible Navier-Stokes equations, the bubbles may be selected
by setting this flag to {\tt False}.
\sifend

\sifitem{Equation}{eq id}
The equation section is used to define a set of equations for a body or set of bodies:
\sifbegin
\sifitem{Navier-Stokes}{Logical} if set to {\tt True}, solve the Navier-Stokes equations.
\sifitem{KE Turbulence}{Logical} If set to {\tt True}, solve the k-epsilon turbulence model
along with Navier-Stokes equations.
\sifitem{Magnetic Induction}{Logical} If set to {\tt True}, solve the magnetic induction equation
along with the Navier-Stokes equations.
\sifitem{Convection}{String [None, Computed, Constant]}
The convection type to be used
in the heat equation, one of: {\tt None}, {\tt Computed}, {\tt Constant}. 
The second choice is used for thermal flows.
\sifend

\sifitem{Body Force}{bf id}
The body force section may be used to give additional force terms for the equations.
\sifbegin
\sifitem{Boussinesq}{Logical} If set true, sets the Boussinesq model on.
\sifitem{Flow BodyForce i}{Real} May be used to give additional body force for
the flow momentum equations, \texttt{i=1,2,3}.
\sifitem{Lorentz Force}{Logical} If set true, triggers the magnetic
field force for the flow mementum equations.

\sifitem{Potential Force}{Logical} If this is set true the force 
used for the electricstatic coupling is activated.  
%
\sifitem{Potential Field}{Real}
 The field to which gradient the external force is proportional to. For example
the electrostatic field.

\sifitem{Potentai Coefficient}{Real} 
The coefficient that multiplies the gradient term. For example, the 
charge density.


\sifend

\sifitem{Initial Condition}{ic id} 
The initial codition section may be used to set initial values for the field
variables. The following variables are active:
\sifbegin
\sifitemnt{Pressure}{Real}
\sifitem{Velocity i}{Real} 
For each velocity component {\tt i}$=1,2,3$.
\sifitem{Kinetic Energy}{Real}
For the k-$\varepsilon$ turbulence model.
\sifitem{Kinetic Energy Dissipation}{Real}
\sifend

\sifitem{Material}{mat id}
The material section is used to give the material parameter values. The
following material parameters may be set in Navier-Stokes equation.
\sifbegin
\sifitemnt{Density}{Real}
The value of density is given with this keyword. The value may be constant,
or variable. For the of compressible flow, the density is computed internally,
and this keyword has no effect.
\sifitem{Viscosity}{Real} 
When using the solidification modelling,
a viscosity-temperature curve must be given. The viscosity must be set
to high enough value in the temperature range for solid material to effectively
set the velocity to zero.
\sifitem{Reference Temperature}{Real} This is the reference temperature for the Boussinesq model
of temperature dependence of density.
\sifitem{Heat Expansion Coefficient}{real} For the Boussinesq model the heat expansion
coefficient must be given with this keyword. Default is 0.0.
\sifitem{Applied Magnetic Field 1,2,3}{Real} An applied magnetic field may be given with these
keywords.
\sifitem{Compressiblity Model}{String}
This setting may be used to set the compressibilty
model for the flow simulations. Currently the setting may be set to either
{\tt Incompressible}, {\tt Perfect Gas} and 
{\tt ArtificialCompressible}. If perfect gas model is chosen 
the settings {\tt Reference Pressure} and {\tt Specific Heat Ratio} must also be given.
The artificial compressibility model may be used to boost convergence in
fluid-structure-interaction cases.
The default value of this setting is {\tt Incompressible}.
\sifitem{Reference Pressure}{Real} with this keyword a reference level of pressure may be given.
This setting applies only if the {\tt Compressiblity Model} is set to
the value {\tt Perfect Gas}.
\sifitem{Specific Heat Ratio}{Real} 
The ratio of specfic heats (in constant pressure
versus in constant volume) may be given with this keyword.
This setting applies only if the {\tt Compressiblity Model} is set to
value {\tt Perfect Gas}. The default value of this setting is $5/3$, which
is the appropriate value for monoatomic ideal gas.
\sifend

For the k-$\varepsilon$ turbulence model the model parameters may
also be given in the material section using the following
keywords
\sifbegin
\sifitemnt{KESigmaK}{Real [1.0]}
\sifitemnt{KESigmaE}{Real [1.3]}
\sifitemnt{KEC1}{Real [1.44]}
\sifitemnt{KEC2}{Real [1.92]}
\sifitemnt{KECmu}{Real [0.09]}
\sifend

Non-newtonian material laws are also defined in material section.
\sifbegin
  \sifitem{Viscosity Model}{String}
The choices are \texttt{power law, carreau, cross, powell ey\-ring} and
\texttt{thermal carreau}.
If none is given the fluid is treated as newtonian.
  \sifitem{Viscosity Exponent}{Real}
	Parameter $n$ in the models power law, Carreau, Cross
  \sifitem{Viscosity Difference}{Real}
  Difference $\Delta \eta$ between high and low shearrate viscosities.
   Ablicable to Carreau, Cross and Powell-Eyring models.
\sifitem{Viscosity Transition}{Real}
   Parameter $c$ in the Carreau, Cross and Powell-Eyring models.
\sifitem{Critical Shear Rate}{Real [0.0]}
Optional parameter $\dot{\gamma}_0$ in power law viscosity model.
\sifitem{Yasuda Exponent}{Real}
Optional parameter $y$ in Carreau model. The default is 2. 
If activated the model is the more generic
Yasuda-Carreau model.
\sifitem{Viscosity Temp Ref}{Real}
Paremeter $T_0$ in the thermal Carreau-Yasuda model. 
\sifitem{Viscosity Temp Exp}{Real}
Paremeter $d$ in the thermal Carreau-Yasuda model. 
\sifend
% 
Porosity is defined by the material properties
\sifbegin
  \sifitem{Porous Media}{Logical}
If this keyword is set \texttt{True} then the porous model will
be active in the material.
  \sifitem{Porous Resistance}{Real}
This keyword may give a constant resistance or also
a orthotropic resistance where the resistance of each velocity component is
given separately.
 \sifend

\sifitem{Boundary Condition}{bc id}
The boundary condition section holds the parameter values for various
boundary condition types. Dirichlet boundary conditions may be
set for all the primary field variables. The one related to Navier-Stokes equation
are
\sifbegin
\sifitem{Velocity i}{Real} 
Dirichlet boundary condition
for each velocity component {\tt i}$=1,2,3$.
\sifitem{Pressure}{Real} 
Absolute pressure.
\sifitem{Normal-Tangential Velocity}{Real}
The Dirichlet conditions for the vector variables may be given in normal-tangential
coordinate system instead of the coordinate axis directed system using the keywords
\sifitem{Flow Force BC}{Logical}
Set to {\tt true}, if there is a force boundary
condition for the Navier-Stokes equations.
\sifitem{Surface Tension Expansion Coefficient}{Real} 
Triggers a tangetial stress boundary condition to be used.
If the keyword {\tt Surface Tension Expansion Coefficient} is given, a linear
dependence of the surface tension coefficient on the temperature is assumed.
Note that this boundary condition is the tangential derivative
of the surface tension coefficient
\sifitem{Surface Tension Coefficient}{Real}
Triggers the same physical model as the previous one except 
no linearity is assumed. The value is assumed to
hold the dependence explicitely. 
\sifitem{External Pressure}{Real} 
A pressure boundary condition directed normal to the surface.
\sifitem{Pressure i}{Real} 
A pressure force in the given direction {\tt i}$=1,2,3$.

\sifitem{Free Surface}{Logical} Specifies a free surface.
\sifitem{Free Moving}{Logical}
Specifies whether the regeneration of mesh is
free to move the nodes of a given boundary when remeshing after moving the free surface nodal
points. The default is that the boundary nodes are fixed.
\sifend

The k-$\varepsilon$ turbulence model also has its own set of boundary condition
keywords (in addition to the Dirichlet settings):
\sifbegin
\sifitem{Wall Law}{Logical} The flag activates the (Reichardts) law of the wall
for the boundary specified.
\sifitem{Surface Roughness}{Real} A measure of the roughness of the surface,
the default is 9.0.
\sifitem{Boundary Layer Thickness}{Real} The thickness of the
laminar layer close to the surface. The element size near the surface outside
the laminar layer into the exponential layer should be close to this value.
\sifend
\sifend


\bibliography{elmerbib}
\bibliographystyle{plain}
