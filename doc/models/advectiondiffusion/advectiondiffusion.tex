\chapter{Advection-Diffusion Equation}
\noindent
\modinfo{Module name}{AdvectionDiffusion}
\modinfo{Module subroutines}{\Idx{AdvectionDiffusionSolver}}
\modinfo{Module authors}{Juha Ruokolainen, Ville Savolainen, Antti Pursula}
\modinfo{Document authors}{Ville Savolainen, Antti Pursula}
\modinfo{Document edited}{Oct 29th 2003}

\section{Introduction}

Advection-diffusion equation (sometimes called diffusion-convection equation)
describes the transport of a scalar quantity or a chemical species by
convection and diffusion. The difference in the nomenclature usually indicates
that an advected quantity does not have an effect on the velocity field of
the total fluid flow but a convected quantity has. Advection-diffusion equation
is derived from the principle of mass conservation of each species in the fluid
mixture. Advection-diffusion equation may have sources or sinks, and several
advection-diffusion equations may be coupled together via chemical reactions.

Fick's law is used to model the diffusive flux. Diffusion may be anisotropic,
which may be physically reasonable at least in solids. If the velocity field
is identically zero, the advection-diffusion equation reduces to the diffusion
equation, which is applicable in solids.

Heat equation is a special case of the advection-diffusion
(or diffusion-convection) equation, and it is described elsewhere in this
manual.

\section{Theory}

\subsection{Governing Equations}

The advection-diffusion equation may, in general, be expressed in
terms of relative or absolute mass or molar concentrations. In Elmer,
when the transported quantity is carried by an incompressible fluid
(or it is diffused in a solid), relative mass concentration
$c_i=C_i/\rho$ for the species $i$ is used ($C_i$ is the absolute mass
concentration in units $\mathrm{kg}/\mathrm{m}^3$, and $\rho$ the
total density of the mixture). We have used the approximation valid
for dilute multispecies flows, i.e., $0\le c_i\ll 1$.  The
advection-diffusion equation is now written as
\begin{equation}
\rho  \left( \frac{\partial c_i}{\partial t}+(\vec v\cdot\nabla) c_i\right) = 
\rho\nabla\cdot(D_i\nabla c_i) + S_i,
\end{equation}
where $\vec v$ is the advection velocity, $D_i$ the diffusion coefficient
and $S_i$ is a source, sink or a reaction term. The diffusion coefficient may
be a tensor.

For a compressible fluid, the concentration should be expressed in absolute
mass units, and the advection-diffusion equation reads
\begin{equation}
\frac{\partial C_i}{\partial t} + (\nabla\cdot\vec{v})C_i  +(\vec v\cdot\nabla) C_i = \nabla\cdot(D_i\nabla C_i) + S_i.
\end{equation}

For a situation, where the quantity is transported through a phase
change boundary, it is convenient to scale the absolute mass
formulation by the respective solubilities of the different
phases. Such a case is for example the surface of a liquid, where the
transported quantity is evaporated into a gaseous material. The
scaled concentration variable satisfies the equilibrium boundary 
condition on the phase
change boundary automatically, and thus the advection-diffusion
equation can be solved for both materials simultaneously. The scaling
is following
\begin{equation}
x_i = \frac{C_i}{C_{i,max}},
\end{equation}
where $x_i$ is the concentration of species $i$ relative to its
maximum solubility in the current material in absolute mass units. The
maximum solubility has to be a constant (temperature independent) for
the absolute mass formulation of the advection-diffusion equation to
remain unchanged.

It is also possible to include temperature dependent diffusion (Soret
diffusion). This introduces an additional term on the right had side
of the equation:
\begin{equation}
\nabla\cdot(\rho D_{i,T}\nabla T),
\end{equation}
where $D_{i,T}$ is the thermal diffusion coefficient of species
$i$. The coefficient $D_{i,T}$ has to be given in the units $m^2/Ks$
regardless of the units used for concentration.

The velocity of the advecting fluid, $\vec v$, is typically calculated by
the Navier-Stokes equation and read in from a restart file. All quantities can
also be functions of, e.g., temperature that is given or solved by the heat
equation. Several advection-diffusion equations for different species $i$ may
be coupled and solved for the same velocity field.

Given volume species sources $S_i$ can be prescribed. They are given
in absolute mass units, i.e., $\mathrm{kg}/\mathrm{m}^3\mathrm{s}$.
If the equation is scaled to maximum solubility, the source term can
be given in absolute mass units, or in scaled units, $S_{i,sc} = S_i /
C_{i,max}$, which is the default.


\subsection{Boundary Conditions}

For each species one can apply either a prescribed concentration or a mass
flux as boundary conditions.

\Idx{Dirichlet boundary condition} reads as
\begin{equation}
c_i=c_{i,b},
\end{equation}
or
\begin{equation}
C_i=C_{i,b},
\end{equation}
depending on the units. If the concentration is scaled to maximum
solubility, the Dirichlet boundary conditions have to be given also in
scaled values, $x_i = C_{i,b}/C_{i,max}$. In all variations, the
boundary value can be constant or a function of time, position or
other variables.

One may specify a mass flux $\vec{\jmath}_i$ perpendicular to the boundary by
\begin{equation}
\vec{\jmath}_i\cdot\vec{n} = -D_i\frac{\partial C_i}{\partial n} = g.
\end{equation}
In relative mass units, this may be written as
\begin{equation}
\vec{\jmath}_i\cdot\vec{n} = -\rho D_i\frac{\partial c_i}{\partial n} = g.
\end{equation}
Thus the units in the flux boundary condition are always
$\mathrm{kg}/\mathrm{m}^2s$ except when the equation is scaled to
maximum solubility. In that case the default is to give flux condition
in scaled units, $g_{sc} = g/C_{i,max}$, although the physical
units are also possible.

The mass flux may also be specified by a mass transfer coefficient
$\beta$ and an external concentration $C_{ext}$
\begin{equation}
-D_i\frac{\partial C_i}{\partial n} = \beta(C_i-C_{i,ext}).
\end{equation}

On the boundaries where no boundary condition is specified, the
boundary condition $g=0$ is applied. This zero flux
condition is also used at a symmetry axis in 2D, axisymmetric or
cylindrical problems.

The equilibrium boundary condition on phase change boundaries under
certain conditions is that the relative amounts of the transported
quantity are equal on both sides of the boundary, 
\begin{equation}
\frac{C_i^{(1)}}{C^{(1)}_{i,max}} = \frac{C_i^{(2)}}{C^{(2)}_{i,max}},
\end{equation}
where the superscripts (1) and (2) refer to different sides of the
boundary. This boundary condition is automatically satisfied if the
equation is scaled with the maximum solubilities $C^{(j)}_{i,max}$.

However, the scaling causes a discontinuity into the mass flux of the
species through the phase change surface. The solver compensates this
effect as long as such a boundary is flagged in the command file by
the user.


\section{Keywords} 

\sifbegin
\sifitem{Simulation}{}
The simulation section gives the case control data:
\sifbegin
\sifitem{Simulation Type}{String} Advection-diffusion equation may be either 
{\tt Transient} or {\tt Steady State}.
\sifitem{Coordinate System}{String} Defines the coordinate system to be used, one of:
{\tt Cartesian 1D}, {\tt Cartesian 2D},~~ {\tt Cartesian 3D},~~ {\tt Polar 2D},~~
 {\tt Polar 3D},~~ {\tt Cy\-lin\-dric},~~ {\tt Cylindric Symmetric}
~~and~~ {\tt Axi Symmetric}.
\sifitem{Timestepping Method}{String} 
Possible values of this parameter are {\tt Newmark} (an additional
parameter {\tt Newmark Beta} must be given), {\tt BDF} ({\tt BDF Order} must be given). Also as a
shortcut to {\tt Newmark}-method with values of {\tt Beta=0.0,0.5, 1.0} the keywords 
{\tt Explicit Euler}, {\tt Crank-Nicolson}, and {\tt Implicit Euler} may be given respectively.
The recommended choice for the first order time integration is the BDF method of order 2.
\sifitem{BDF Order}{Integer}
Value may range from 1 to 5.
\sifitem{Newmark Beta}{Real} Value in range from 0.0 to 1.0. The value 0.0 equals to
the explicit Euler integration method and the value 1.0 equals to the implicit Euler method. 
\sifend

\sifitem{Solver}{solver id}
The solver section defines equation solver control variables. Most of the possible
keywords -- related to linear algebra, for example -- are common for all the solvers and are 
explained elsewhere.
\sifbegin
\sifitem{Equation}{String [Advection Diffusion Equation Variable\_name]}
The name of the equation, e.g., {\tt Advection Diffusion Equation Oxygen}.
\sifitem{Variable}{String Variable\_name}
The name of the variable, e.g., {\tt Oxygen}.
\sifitem{Procedure}{File "AdvectionDiffusion"\ "AdvectionDiffusionSolver"}
The name of the file and subroutine.
\sifitem{Nonlinear System Convergence Tolerance}{Real}
The criterion to
terminate the nonlinear iteration after the relative change of the norm of the field variable
between two consecutive iterations $k$ is small enough
$$
 ||u_k-u_{k-1}|| < \epsilon ||u_k||,
$$
where $\epsilon$ is the value given with this keyword, and $u$ is either $c_i$
or $C_i$.
\sifitem{Nonlinear System Max Iterations}{Integer}
The maximum number of nonlinear iterations the solver is allowed to do.
\sifitem{Nonlinear System Relaxation Factor}{Real}
Giving this keyword triggers the use
of  relaxation in the nonlinear equation solver.
Using a factor below unity is sometimes required to achieve convergence of the nonlinear system.
A factor above unity might speed up the convergence. Relaxed variable is defined as follows:
$$
 u^{'}_k = \lambda u_k + (1-\lambda) u_{k-1},
$$
where $\lambda$ is the factor given with this keyword. The default value for the relaxation factor
is unity.
\sifitem{Steady State Convergence Tolerance}{Real}
With this keyword a equation specific steady state or coupled system
convergence tolerance is given.
All the active equation solvers must meet their own tolerances for their
variable $u$ before the 
whole system is deemed converged.
The tolerance criterion is:
$$
 ||u_i-u_{i-1}|| < \epsilon ||T_i||,
$$
where $\epsilon$ is the value given with this keyword.
\sifitem{Stabilize}{Logical} 
If this flag is set true the solver will use stabilized finite element method
when solving the advection-diffusion equation with a convection term.
If this flag is set to
{\tt False}, RFB (Residual Free Bubble) stabilization is used instead (unless
the next flag {\tt Bubbles} is set to {\tt False} in a problem with Cartesian
coordinate system).
If convection dominates, some form of stabilization must be used in order to succesfully solve the equation.
The default value is {\tt False}.
\sifitem{Bubbles}{Logical}
There is also a residual-free-bubbles formulation of the stabilized finite-element
method. It is more accurate and does not include any ad hoc terms. However, it may
be computationally more expensive. The default value is {\tt True}.
If both {\tt Stabilize} and {\tt Bubbles} or set to {\tt False}, no stabilization
is used. This choice may be enforced in a problem with Cartesian coordinates,
but the results might be nonsensical. Both
{\tt Stabilize} and {\tt Bubbles} should not be set to {\tt True}
simultaneously.
\sifend

\sifitem{Equation}{eq id}
The equation section is used to define a set of equations for a body or set of bodies.
\sifbegin
\sifitem{Advection Diffusion Equation Variable\_name}{Logical} If set to {\tt True}, solve the advection-diffusion equation.
\sifitem{Convection}{String}
The type of convection to be used
in the advection-diffusion equation, one of: {\tt None}, {\tt Computed}, {\tt Constant}.
\sifitem{Concentration Units}{String}
If set to {\tt Absolute Mass}, absolute mass units are used for concentration.
Recommended for a compressible flow. Also possible to select {\tt Mass
To Max Solubility} which causes the absolute mass formulation of the 
equation to be scaled by the maximum solubilities of each material.
\sifend

\sifitem{Body Forces}{bf id} 
The body force section may be used to give additional force terms for the equations.
The following keyword is recognized by the solver:
\sifbegin
\sifitem{Variable\_name Diffusion Source}{Real}
An additional volume source for the advection-diffusion equation may be given
with this keyword. It may depend on coordinates, temperature and other variables, such as concentration of other chemical species, and thus describe a source,
a sink or a reaction term. Given in absolute mass units or, in case of
scaling, in the scaled units.
\sifitem{Physical Units}{Logical True}
With this keyword, the source term can be given in absolute mass units
regardless of scaling.
\sifend


\sifitem{Initial Condition}{ic id}
The initial condition section may be used to set initial values for the
concentration $c_i$, $C_i$ or $x_i$.
\sifbegin
\sifitemnt{Variable\_name}{Real}
\sifend

\sifitem{Material}{mat id}
The material section is used to give the material parameter values. The
following material parameters may be effective when advection-diffusion
equation is solved.
\sifbegin
\sifitem{Convection Velocity i}{Real} 
Convection velocity {\tt i}$=1,2,3$ for the constant convection model.
\sifitem{Density}{Real}
The value of density of the transporting fluid 
is given with this keyword. The value may be constant,
or variable. For compressible flow, the density of the transporting fluid
is computed internally, and this keyword has no effect.
\sifitem{Compressibility Model}{String} This setting may be used to set the
compressibility
model for the flow simulations. Choices are {\tt Incompressible} and {\tt Perfect Gas}.
If set to the latter, the density is calculated from the ideal gas law.
Then also the settings {\tt Reference Pressure}, {\tt Specific Heat Ratio}
and {\tt Heat Capacity} must be given.
\sifitem{Reference Pressure}{Real} With this keyword a reference level of pressure may be given.
\sifitem{Specific Heat Ratio}{Real} The ratio of specific heats (in constant pressure
versus in constant volume) may be given with this keyword.
The default value of this setting is $5/3$, which
is the appropriate value for monoatomic ideal gas.
\sifitem{Heat Capacity}{Real} For the compressible flow, specific heat
in constant volume.
\sifitem{Variable\_name Diffusivity}{Real}
The diffusivity $D$ given by, e.g., {\tt Oxygen Diffusivity}. Can be a constant
or variable. For an anisotropic case, may also be a tensor $D_{ij}$.
\sifitem{Variable\_name Soret Diffusivity}{Real}
The thermal diffusivity coefficient $D_T$ given by, e.g., {\tt Oxygen}
{\tt Soret Diffusivity}. 
Can be a constant or variable.
\sifitem{Variable\_name Maximum Solubility}{Real}
The maximum solubility of the species in absolute mass units. 
Has to be a constant value.
\sifend


\sifitem{Boundary Condition}{bc id}
In advection-diffusion equation we may set the concentration directly 
by Dirichlet boundary conditions or use mass flux condition.
The natural boundary condition is zero flux condition.
\sifbegin
\sifitemnt{Variable\_name}{Real}
\sifitemnt{Mass Transfer Coefficient}{Real}
\sifitem{External Concentration}{Real}
These two keywords are used to define flux condition that depends on
the external concentration and a mass transfer coefficient. This
condition is only applicable to absolute mass formulation of the
equation (see keywords for Equation block).
\sifitem{Variable\_name Flux}{Real} A user defined mass flux term in absolute
mass units or, in case of scaling, in the scaled units.
\sifitem{Physical Units}{Logical True}
With this keyword, the flux boundary condition can be given in 
absolute mass units regardless of scaling. Note that this keyword does
NOT affect the Dirichlet boundary condition nor the mass transfer
coefficient bc.
\sifitem{Variable\_name Solubility Change Boundary}{Logical True}
This keyword marks the boundary over which the maximum solubility
changes. Has to be present for the mass flux continuity to be
preserved.
\sifitem{Normal Target Body}{Integer bd id}
In a solubility change boundary, this keyword can be used to control 
on which side the mass flux compensation is done. Basically, this can
be done on either side but there can be some effect on
the accuracy or on the speed of the solution. Recommended is to give as
normal target the body with less dense mesh, or the direction of
average species transport. If normal target body is not specified, the
material with smaller density is used.
\sifend
\sifend
