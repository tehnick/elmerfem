\Chapter{Free Surface with Constant Flux}

\modinfo{Module name}{\Idx{FreeSurfaceReduced}}
\modinfo{Module subroutines}{FreeSurfaceReduced}
\begin{versiona}
\modinfo{Module authors}{Peter R�back}
\modinfo{Document authors}{Peter R�back}
\modinfo{Document edited}{August 5th 2002}


\section{Introduction}

The determination of free surface is often an essential part of solving
a fluid dynamics problem. Usually the surface is found by solving a 
free surface equation resulting from force balance, 
or by finding the free surface from zero flux condition.
In some extreme cases both of these methods were found to fail and 
therefore an alternative approach was taken. The method can 
only be applied to stationary 2D or axisymmetric flows where the total flux 
is conserved. This is the case, for example, in many \Idx{coating} and 
\Idx{drawing} processes.


\section{Theory}

\index{mass conservation}
The determination of the free surface takes use of the 
conservation of mass. If the flow is stationary the mass flux through all 
planes cutting the flow must be same.
In the following we concentrate on the axisymmetric case which has more 
applications than the 2D case. 

In the axisymmetric case the mass flux is obtained from 
\begin{equation}
  f(R,z) = \int_{R_0}^{R} (\vec{u}\cdot\vec{n}) \, r ds.
\end{equation}
The free surface is set by finding 
a surface profile $R(z)$ such that the integral is
constant for all nodes on the surface, or 
\begin{equation}
  f(R,z_j) = f(R_1,z_1) \ \ \ \ \forall j \in  [1,M].
\end{equation}
Note that the factor $2\pi$
has been consistently omitted since it has no bearing
to the shape of the free surface.

The subroutine uses simple heuristics to determine the direction 
of the flow on the free surface. The first upwind node $z_1$ on the free surface is 
assumed to be fixed and the corresponding flux is $f_1$.
The new radius is set approximately by assuming that the 
added or removed flow has the same velocity as the 
velocity on the surface.
Then the corrected radius is found from
\begin{equation}
  u_n R^{(m)} \, d R^{m} = f(R^{(m)},z) - f(R_1,z_1)
\end{equation}
or
\begin{equation}
  R^{(m+1)} = R^{(m)} + \frac{f(R^{(m)},z) - f(R_1,z_1)}{u_n R^{(m)}}.
\end{equation}
After the new profile is being found the 
element nodes are moved to the new positions.
The nodes that are not on the surface may be mapped in 
many different ways. The straight-forward strategy is to use linear 1D mapping.
Also more generic 2D mapping may be used.
 
The free surface and the fluid flow must 
be consistent and therefore the system must 
be solved iteratively.
When convergence of the coupled system has been obtained 
the suggested $d R$ vanishes and the free surface solver 
does not affect the solution.

Sometimes the free surface solver overshoots and therefore it may
be necessary to use relaxation to suppress the large 
changes of the solution.

Note that the free surface solver is simple based on mass conservation. No
forces are applied on the free surface. If surface tension needs to be taken 
into account it may be done while solving the Navier-Stoke equation.


\section{Applicable cases and limitations}

The method has some limitations which are inherent of the method:
\begin{itemize}
\item Limited to steady-state simulations.
\item Limited to 2D and axisymmetric cases.
\item If there is back-flow within the free surface flow the correctness of the 
solution is not guaranteed.
\end{itemize}
Some limitations result from the current implementation:
\begin{itemize}
\item The free surface must be oriented so that the flow is on its negative side.
\item There may be several free surfaces of this type but they must be directed
	the same way.
\item The line integral from $R_0$ to $R$ may cause some difficulties in 
	unstructured meshes. Therefore structured meshes are favored.
\item At the moment density is assumed to be constant and
 	therefore only incompressible fluids may be considered.
\end{itemize}



\section{Keywords}
\end{versiona}

\sifbegin
\sifitemnt{Solver}{solver id}
\sifbegin
\sifitemnt{Equation}{String "Free Surface Reduced"}
\sifitem{Variable}{String dx}
The change in the free surface coordinate.
This may be of any name as far as it is used consistently also elsewhere.
\sifitem{Variable DOFs}{Integer 1}
Degrees of freedom for the free surface coordinate.
\sifitem{Procedure}{File "FreeSurfaceReduced"\ "FreeSurfaceReduced"}
The following four keywords are used for output control.
\sifitem{Perform Mapping}{Logical}
If this keyword is {\tt True} the coordinate 
mapping is done locally by using linear 1D mapping.
This is also the default. 
Also 2D mapping is possible by using a separate 
mesh update solver. Then the keyword should be set to \texttt{False}.
\sifitem{Nonlinear System Relaxation Factor}{Real}
The changes in the free surface may be relaxed. The 
default is no relaxation or value 1.0

\sifitem{Nonlinear System Convergence Tolerance}{Real} 
This keyword gives a criterion to
terminate the nonlinear iteration after the maximum change in the 
free surface coordinate is small enough
\begin{equation}
 \max || d R / ( R - R_0 ) ||  < \epsilon  \nonumber
\end{equation}
where $\epsilon$ is the value given with this keyword.
\sifend

\sifitemnt{Boundary Condition}{bc id}
\sifbegin
\sifitem{Free Surface Reduced}{Logical}
Must be set to {\tt True} for the free surface when the solver is used.
The boundary must be simply continuous.
\sifitem{Free Surface Number}{Integer}
If more than one free surface of the reduced type is present simultaneously
they must somehow be separated. This keyword is for that purpose.
The surfaces should be ordered from 1 to the number or free surfaces.
Value 1 is also the default if the surface is active.
Note that free surfaces with different numbers should be aligned the same way and should 
not touch each other.
\sifitem{Free Surface Bottom}{Logical}
If this flag is free it sets the lower boundaries of integration 
when solving for the free surface. Note that this surface should not touch any of the 
free surfaces. A free surface is automatically a lower boundary for another 
free surface.
\sifend
If mapping is not performed within the solver also boundary conditions
for the mapping are required. Surface tension may be taken into account 
while solving the Navier-Stokes equation. The proper keywords for 
activating the surface tension are explained in the manual of the Navier-Stokes solver.
\sifend



%\bibliography{elmerbib}
%\bibliographystyle{plain}

