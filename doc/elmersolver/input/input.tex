\chapter{Solver Input File}
\noindent
                                                                                                                 
\section{Introduction}
                                                                                                                 
The solver input (.sif) file describes the
computational case for the solver.  The
input file is used to give the solver
the mesh directory,  material parameters, initial
conditions, boundary conditions, etc.
One  may add his/her own keywords to the file,
which solvers may ask using special runtime routines.
The different parameters given in input file may
depend on solution fields, and may be given as constants, tabulated
values,  MATC  functions, or Fortran  functions. MATC is an
interpreted language, which doesn't need any
additional compilation step, or a compiler.

\section{A Sample input file}

The input file consists of different sections
\begin{itemize}
\item Header section
\item Simulation section
\item Body n - sections
\item Initial Condition n - sections
\item Body Force n - sections
\item Material n - sections
\item Equation n - sections
\item Boundary Condition n - sections
\end{itemize}

Each section consists of the name of the section, a number of
keyword-value pairs, and a row containing the word 'End'.
A sample input file header might be 

\ttbegin
Check Keywords "Warn"
Header
  Mesh DB "." "1d"
End
\ttend

Here only the placing of the mesh files is given. The simulation
section contains several fields that concern the case as a whole:

\ttbegin
Simulation
  Coordinate System = "Cartesian 1D"
  Coordinate Mapping(3) = 1 2 3
  Simulation Type = Steady
  Output Intervals(1) = 1
  Steady State Max Iterations = 1
  Post File = "1dheat.ep"
  Output File = "1dheat.dat"
End
\ttend

The computational domain might consist of several different subdomains or
'bodies', for each body the equation definition section, initial condition
section, body force section and material definition section numbers are
given. These sections are described later.

\ttbegin
Body
  Equation = 1
  Material = 1
  Body Force = 1
End
\ttend

The body force definition is used to give the volume forces, if any,
for the equation solvers. Here a source term for the Poisson equation
is given as constant 1:

\ttbegin
Body Force 1
  Source = Real 1
End
\ttend

The equation section defines the set of equation solvers for a subdomain
of the whole computational  domain to be activated.  Note that previously
we attached the equation definition to a body definition. This time only
one equation solver is activated.

\ttbegin
Equation 1
  Active Solvers(1) = 1
End
\ttend

The solver sections are used to define the equations solver and all the
solver dependent parameters, such as linear system solver options, and
the file name and name of the solver procedure, the name of the solver
variable, etc.

\ttbegin
Solver 1
 Equation = "Poisson"
 Variable = "Potential"
 Variable DOFs = 1
 Procedure = "Poisson" "PoissonSolver"
 Linear System Solver = "Direct"
 Steady State Convergence Tolerance = 1e-06
End
\ttend

Boundary condition sections give the boundary conditions for the different
equations. The boundary condition is mapped to boundaries of the mesh by
giving the keyword Target Boundaries. Below the meaning is that Boundary
Condition 1 is attached to boundaries 1, and 2:

\ttbegin
Boundary  Condition 1
  Target Boundaries(2) = 1 2
   Potential = Real 0
End
\ttend

\section{Keyword syntax}

A keyword is given a value with the equals sing:

\ttbegin
Density  = 1000 ! water
\ttend

If the keyword is not known to the Elmer Solver the type of the
value  must also be given

\ttbegin
My Parameter = Real 1000
\ttend

Valid value types are
\begin{itemize}
\item Real
\item Integer
\item Logical
\item String
\item File
\end{itemize}
Most parameters may depend on field variables defined in the current run:

\ttbegin
Density = Variable Temperature)
  MATC "1000*(1-1.0e-4*(tx-273))"
\ttend

This would mean, that the value of Density is dependent on Tempereture as:
\begin{equation}
  \rho =  \rho_0(1-\beta(T-T_0)),
\end{equation}
with $\rho_0=1000, beta=10^{-4}$ and $T_0=273$.
The value of the independent variable is known as "tx" in the MATC language.
In addition the variable "st" will contain the current steady iteration
number or simulation time, if the run is steady or transient respectively.
If the independent variable has more than one component, the variable "tx"
will contain all the component values in (tx(0),tx(1),...,tx(n-1), where n
is the number of components of the independent variable. In the same manner
a user Fortran 90 function may be substituted in  place of the MATC function

\ttbegin
Density  = Variable Temperature
  Procedure "filename" "proc"
\ttend

Where the file "filename" should contain the shareable .so (Unix) or .dll
(Windows) code for the user function whose name is "proc".  The call interface
for the Fortran function is as follows

\ttbegin
FUNCTION proc( Model, n, T ) RESULT(dens)
  USE DefUtils)
  IMPLICIT None
  TYPE(Model_t) :: Model)
  INTEGER :: n
  REAL(KIND=dp) :: T, dens

  dens = 1000*(1-1.0d-4(T-273.0d0))
END FUNCTION proc
\ttend

The Model structure contains the whole definition of the Elmer run, and may be
used to obtain field variable values, nodal coordinates, etc. The argument n is
the node number to be processed, and t is the value of the independent variable
at the nodal point. The routine should return the value  of the dependent
variable. Another example might be

\ttbegin
Density = Variable Coordinate
 Procedure "filename" "proc"
\ttend

Now the argument t would be an array of three values, which would contain the
x,y, and z  coordinates of the current nodal point. One  more way to give the
dependency is give a tabulated  pair of values:

\ttbegin
Density = Variable Temperature
  Real
   0 900
   273 1000
   300 1020
   400 1000
End
\ttend

This means, that when Temperature is 0, Density gets value 900, and is
linearily interpolated between 0-273 between values 900 and 1000, etc.
If the value of the independent variable is outside the range of the
tablulated values,the first or the last of the linear equations from the
tabulated  values is used to extrapolate the value of the dependent
variable. A keyword may be defined to be a array insted of a scalar
value, the array may be defined by giving the size of the array as in

\ttbegin
Active Solver(2) = 1 2
Target Boundaries(3) = 2 4 5
My Parameter Array(3,3) = Real 1 2 3 \keno
                               4 5 6 \keno
                               7 8 9 

\ttend

The components of the array valued variables (defined for example in solver
section) are given a name with the component number attached, i.e. if the
solver section has a definition of  a field variable

\ttbegin
Variable = "Displ"
Variable DOFs = 3
\ttend

then Elmer Solver knows in addition to the vector Displ three scalar fields
'Displ 1', 'Displ 2' and 'Displ 3'.   These latter field names may be used
to set initial conditions, boundary conditions for the components and
used as independent variables  in parameter definitions,etc.

\bibliography{elmerbib}
\bibliographystyle{plain}
