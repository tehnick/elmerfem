\appendix
\chapter{Structure of Elmer Simulation}

\section{Information flow}

The Elmer environment includes several different programs that 
communicate only by files of specific files. Often there is
more than one tool that may be used to make the files and 
therefore there is also several ways how a problem may be solved.

% 1) save PowerPoint figure by Acrobat Distiller to myfile.pdf
% 2) pdftops myfile.pdf
% 3) ps2epsi myfile.ps myfile.eps
% 4) define BoundingBox by ghostview
% 5) use epsfig to include the figure

\begin{figure}[tbhp]
\vspace{10mm}
\begin{center}
\epsfig{figure=elmer-chart.eps,width=10cm,angle=270,
bbllx=130,bblly=130,bburx=580,bbury=680,clip=1}
\caption{Work flow in the Elmer environment}
\end{center}
\end{figure}


\section{Types of files}

Elmer may use a number of different files in the solution of the
different cases.

Here is some information about the files Elmer used and created during the
different cases. These include files with suffix
.egf, .grd, .sif, and .ep.
\begin{itemize}
\item The Elmer geometry file format (.egf file) contains the structures
      geometry. ElmerFront uses this to get the information of the geometry.
\item The ElmerGrid geometry format (.grd file) contains the
      geometry as well as the mesh definition. The computational mesh
      is done with the Elmergrid software, look for a specific manual
      on that for more details.
\item The ElmerSolver Input File format (.sif file) is usually written by
      ElmerFront. It contains all the data that is used by the Solver.
      By modifying this  file it is possible to variate the Solver input
      easily. This file may also be generated from scratch following
      the Model Manuals and using a text editor.
\item The ElmerPost File format (.ep file) is written by the Solver.
      It contains the solution data. By opening the .ep file in Elmer
      Post it is possible to view the results without needing to solve
      the whole problem again.
\item The file with name ELMERSOLVER\_STARTINFO should be
      situated in the directory were the ElmerSolver is lauched.
      The file just includes the name of the command file (.sif) and
      the number of processors.
\end{itemize}


