\chapter{Solving eigenvalue problems}
\noindent

\section{Introduction}

Eigenvalue problems form an important class of numerical problems,
especially in the field of structural analysis. Also some other
application fields may have eigenvalue problems, such as those in
density functional theory. This manual, however, introduces eigenvalue
computation in Elmer using terminology from elasticity.

Several different eigenvalue problems can be formulated in
elasticity. These include the ``standard'' generalized eigenvalue
problems, problems with geometric stiffness or with damping, as well
as stability (buckling) analysis. All of the aforementioned problems
can be solved with Elmer. The eigenproblems can be solved using
direct, iterative or multigrid solution methods.


\section{Theory}

The steady-state equation for elastic deformation of solids may be
written as
\begin{equation}
-\nabla\cdot\tau = \vec{f},
\end{equation}
where $\tau$ is the stress tensor. When considering eigen frequency
analysis, the force term $\vec{f}$ is replaced by the inertia term,
\begin{equation}
-\nabla\cdot\tau = \rho\frac{\partial^2 \vec{d}}{\partial t^2},
\end{equation}
where $\rho$ is the density.

The displacement can now be assumed to oscillate harmonically with the
eigen frequency $\omega$ in a form defined by the eigenvector
$\vec{d}$. Inserting this into the above equation yields
\begin{equation}
-\nabla\cdot\tau(\vec{d}) = 
-\omega^2\rho\vec{d},
\end{equation}
or in discretized form
\begin{equation}
\label{gen_eigen}
Ku = -\omega^2Mu,
\end{equation}
where $K$ is the stiffness matrix, $M$ is the mass matrix, and $u$ is
a vector containing the values of $\vec{d}$ at discretization
points. The equation~\ref{gen_eigen} is called the generalized
eigenproblem.

Including the effects of pre-stresses into the eigenproblem is quite
straightforward. Let us assume that there is a given tension field
$\sigma$ in the solid. The tension is included by an extra term in the
steady-state equation
\begin{equation}
\label{pre_eigen}
-\nabla\cdot\tau -\nabla\cdot(\sigma\nabla u) = \vec{f}.
\end{equation}
The pre-stress term includes a component $K_G$ to the stiffness matrix
of the problem and thus the eigenvalue equation including pre-stresses
is
\begin{equation}
(K+K_G)u = -\omega^2Mu.
\end{equation} 

The pre-stress in Elmer may be a known pre-tension, due to external
loading or due to thermal stress, for example. The stress tensor
containing the pre-stresses $\sigma$ is first computed by a
steady-state analysis and after that the eigenvalue problem is
solved. It should be noted though that the eigenvalue problem in a
pre-stressed state is solved using first order linearization, which
means that the eigenvalues are solved about the non-displaced
state. If the pre-loading influences large deformations the
eigenvalues are not accurate.

The eigenvalue problem with pre-stresses may be used to study the
stability of the system. Some initial loading is defined and a
pre-stress tensor $\sigma$ is computed. This tensor is then multiplied
by a test scalar $\lambda$. The critical load for stability, or
buckling, is found by setting the force on the right hand side of the
equation~\ref{pre_eigen} equal to zero. The problem then is to solve
$\lambda$ from
\begin{equation}
Ku = -\lambda K_G u,
\end{equation}
which again is formally an eigenvalue problem for the test
parameter. The critical loading is found by multiplying the given test
load with the value of $\lambda$. In other words, if $\lambda > 1$ the
loading is unstable.

\subsection{Damped eigenvalue problem}

Finally, let us consider the damped eigenproblem, also called
quadratic eigenvalue problem. In this case there is a force component
proportional to the first time derivative of the displacement in
addition to the inertia term
\begin{equation}
-\nabla\cdot\tau = -\delta\frac{\partial \vec{d}}{\partial t} +
 \rho\frac{\partial^2 \vec{d}}{\partial t^2},
\end{equation}
where $\delta$ is a damping coefficient. The problem is transformed
into a more suitable form for numerical solution by using a new
variable $\vec{v}^{\prime}$ defined as
$\vec{v}^{\prime}=\frac{\partial \vec{d}}{\partial t}$. This yields
\begin{equation}
-\nabla\cdot\tau = -\delta\vec{v}^{\prime} +
\rho\frac{\partial \vec{v}^{\prime}}{\partial t}.
\end{equation}

Working out the time derivatives and moving into the matrix form, the
equation may be written as
\begin{equation}
Ku=-Dv+i\omega Mv,
\end{equation}
or,
\begin{equation}
\label{damp_eigen}
-i\omega\left(\begin{array}{cc}
I & 0 \\ 0 & M \end{array}\right) \left(\begin{array}{c}
u \\ v \end{array} \right) = \left(\begin{array}{cc}
0 & I \\ -K & -D \end{array}\right) \left(\begin{array}{c}
u \\ v\end{array} \right),
\end{equation}
where $i$ is the imaginary unit, $D$ is the damping matrix, and $v$ a
vector containing the values of $\vec{v}^{\prime}$ at the
discretization points. Now the damped eigenproblem is transformed into
a generalized eigenproblem for complex eigenvalues.

Finally, to improve the numerical behavior of the damped eigenproblem,
a scaling constant $s$ is introduced into the definition
$s\vec{v}^{\prime}=s\frac{\partial \vec{d}}{\partial t}$. In the
matrix equation~\ref{damp_eigen} this influences only the identity
matrix blocks $I$ to be replaced by $sI$. Good results for numerical
calculations are found when 
\begin{equation}
s = ||M||_\infty = \max |M_{i,j}|.
\end{equation}



\section{Keywords related to eigenvalue problems}

An eigenvalue analysis in Elmer is set up just as the corresponding
steady-state elasticity analysis. An eigenvalue analysis is then
defined by a few additional keywords in the Solver section of the sif
file. The solver in question can be linear elasticity solver called
Stress Analysis, linear plate elasticity solver, or even nonlinear
elasticity solver, though the eigen analysis is, of course,
linear. 

Many of the standard equation solver keywords affect also the eigen
analysis, {\em e.g.} the values given for Linear System Solver and
Linear System Iterative Method in case of an iterative solver. More
information about these settings is given in this Manual under the
chapter concerning linear system solvers. The specific keywords for
eigen analysis are listed below

\sifbegin 
\sifitem{Eigen Analysis}{Logical} 
Instructs Elmer to use eigensystem solvers. Must be set to True
in all eigenvalue problems.
\sifitem{Eigen System Values}{Integer}
Determines the number of eigen values and eigen vectors computed.

\sifitem{Eigen System Select}{String}
This keyword allows the user to select, which eigenvalues are
computed. The allowable choices are
\begin{itemize}
\item Smallest Magnitude
\item Largest Magnitude
\item Smallest Real Part
\item Largest Real Part
\item Smallest Imag Part
\item Largest Imag Part
\end{itemize}
Smallest magnitude is the default.
 
\sifitem{Eigen System Convergence Tolerance}{Real}
The convergence tolerance for iterative eigensystem solver. The
default is 100 times Linear System Convergence Tolerance.
\sifitem{Eigen System Max Iterations}{Integer}
The number of iterations for iterative eigensystem solver. The
default is 300.
\sifitem{Eigen System Complex}{Logical}
Should be given value True if the eigen system is complex, {\em i.e.}
the system matrices are complex. Not to be given in damped eigen value
analysis.

\sifitem{Geometric Stiffness}{Logical}
Defines geometric stiffness (pre-stress) to be taken into account in
eigen analysis. This feature is only available with linear bulk
elasticity solver.
\sifitem{Stability Analysis}{Logical}
Defines stability analysis. This feature is only available with linear
bulk elasticity solver.

\sifitem{Eigen System Damped}{Logical}
Defines a damped eigen analysis. Damped eigen analysis is available
only when using iterative solver.
\sifitem{Eigen System Use Identity}{Logical}
If True defines the relation displacement and its derivative to be
$s\vec{v}^{\prime} = s\frac{\partial \vec{d}}{\partial t}$. The other
possibility is to use $Mv=i\omega Mu$. The default is True.

\sifend


\section{Constructing matrices M and D in Solver code}

In eigen analysis the mass matrix $M$ and the damping matrix $D$ have
to be separately constructed. Usually in Elmer the different matrices
are summed into a single matrix structure, since the final linear
equation is of the form $Ax=b$, and there is no need for separate
values of the mass matrix and the stiffness matrix. 

The matrix is represented in Elmer using compressed row storage (CRS)
format, as explained in chapter about Linear system solvers. The
matrix structure holds also vectors for the values of the mass and
damping matrices
\ttbegin
  TYPE Matrix_t
     ...
    REAL(KIND=dp),  POINTER :: MassValues(:), DampValues(:)
     ...
  END TYPE Matrix_t
\ttend
These arrays use the same {\tt Rows} and {\tt Cols} tables than the
normal {\tt Values} array.

The mass and damping matrices are constructed elementwise in a
similar manner as the stiffness matrix. After each element the local
contributions are updated to the equation matrices by the following
subroutine calls
\ttbegin
  CALL DefaultUpdateEquations( STIFF, FORCE )

  IF ( Solver % NOFEigenValues > 0 ) THEN
    CALL DefaultUpdateMass( MASS )
    CALL DefaultUpdateDamp( DAMP )
  END IF
\ttend

In this segment of code the variables {\tt STIFF}, {\tt MASS}, {\tt
DAMP} and {\tt FORCE} store the local values of the stiffness matrix,
the mass matrix, the damping matrix, and the right hand side of the
equation, respectively. The integer {\tt NOFEigenValues} if the {\tt
Solver} data structure gives the number of eigen values
requested. Here it is used as an indicator of whether the mass and
damping matrices need to be constructed.

The eigenvalues and eigenvectors are stored in the arrays {\tt Solver
\% Variable \% EigenValues} and {\tt Solver \% Variable \% EigenVectors},
% respectively. The arrays are defined as complex variables:
\ttbegin
  TYPE Variable_t
     ...
    COMPLEX(KIND=dp), POINTER :: EigenValues(:)
    COMPLEX(KIND=dp), POINTER :: EigenVectors(:,:)
     ...
  END TYPE Matrix_t
\ttend
and the eigenvector corresponding to the eigenvalue {\tt i}
is found in {\tt Solver \% Variable \% EigenVectors(i,:)}.

\bibliography{elmerbib}
\bibliographystyle{plain}

