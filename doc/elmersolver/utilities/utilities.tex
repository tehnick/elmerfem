\chapter{Miscallenous options}
\noindent

\section{Solver activation}

There is a large number of different ways how solvers need to be activated and deactivated.
Mostly there needs are related to different kinds of multiphysical coupling schemes.
In the solver section one may give the following keywords.
\sifbegin
  \sifitem{Exec Solver}{String}
  The options are \texttt{never, always, before timestep, after timestep, bofore all, after all}. 
  If nothing else is specified the solver is called every time in its order of appearance.
  \sifitem{Exec Interval}{Integer}
  This keyword gives an interval at which the solver is active. At other intervals the solver is not 
  used. 
\sifend


%Variable = a[b:n...]
%ja n on 2 tai 3 niin tulee .ep fileeseen vector-m��reell�.
%eih�n tuota ehk� ihan aina haluasi ;-)

%Laitoin sellaisen option, ett� voi antaa ratkaisun komponenteille nimet:
%Variable = flow[Velo:2 Pres:1]

%esimerkiksi. Siis oletusnimien
%Flow 1, Flow 2, Flow 3
%sijaan tulee muuttujalistaan komponenttinimet
%Velo 1, Velo 2 ja Pres

%Variable DOFs avainsanaa, tai -dofs optiota ei tarvitse antaa, eik�
%kannata ettei mene vahingossa sikinsokin...




\bibliography{elmerbib}
\bibliographystyle{plain}

