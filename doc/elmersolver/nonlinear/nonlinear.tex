\chapter{Nonlinear system options}
\noindent

\section{Introduction}

Numerical methods in linear algebra are usually intended for the solution of linear
problems. However, there are many problems which are not linear in nature. The nonlinearity 
may a intrinsic characteristics of the equation, such as is the case with intertial forces
in the Navier-Stokes equation. The nonlinerity might also a result of nonlinear material parameters
that depend on the solution. What ever the reason for nonlinearity the equations in Elmer are 
always first linearized to the form
\begin{equation}\label{ModelLinearSystem}
A(u_{i-1})u_i = b(u_{i-1}),
\end{equation}
where $i$ refers to the iteration cycle. 

How the equations are linearized varies from solver toanother. For example, in the Navier-Stokes 
solver there are tow different methods -- the Picard linearization and the Newton linearization that 
may be used. Also a hybrid scheme where the Picard type of scheme is switched to the 
Newton kind of scheme when certain criteria are met is available. Therefore this section will
not deal with the particular linearization technique of different solver but tries to 
give some light to the generic keywords that are available. Some keywords may also be defined
in the Models Manual related to particular solvers. 

In multiphysical simulations there are also a number of keywords related to 
the solution of coupled systems. Basically one may may define how many times a 
system of equations is solved repeatedly at maximum and how what are the convergence
criteria of the individual solvers that must be met simulataneously.


\section{Keywords related to solution of nonlinear systems}

These keywords are located in the Solver section of each solver, if requited at all.
\sifbegin
\sifitem{Nonlinear System Convergence Measure}{String} 
The change of solution between two consecutive iterations may be estimated by a number 
of different measures which are envoked by values \texttt{norm}, \texttt{solution} and
\texttt{residual}. The default way of checking for convergence is to test the 
change of norm
\begin{equation}
  \delta =  2*||u_i|-|u_{i-1}|| / (|u_i|+|u_{i-1}|).
\end{equation}
This measure is rather liberal since the norm of two solutions may be the same
even though the solutions would not. Therefore it is often desirable to look at the 
norm of change,
\begin{equation}
  \delta = 2*|u_i-u_{i-1}| / (|u_i|+|u_{i-1}|).
\end{equation}
The third choice is to use a backward norm of the residual 
where the old solution is used with the new
matrix. 
\begin{equation}
  \delta = |Ax_{i-1}-b| / |b|.
\end{equation}
In the current implementation this norm lags one step behind and therefore always 
performs one extra iteration.
%
\sifitem{Nonlinear System Norm Degree}{Integer}
The choice of norms used in the evaluation of the convergence measures is not self evident.
The default is the $L2$ norm. This keyword may be used to replace this by $Ln$ norm 
where value $n=0$ corresponds to the infinity (i.e. maximum) norm.
%
\sifitem{Nonlinear System Norm Dofs}{Integer}
For vector valued field variables by default all components are used in the computation of
the norm. However, sometimes it may be desirable only to use some of them. This keyword 
may be used to give the number of components used in the evaluation. For example, in the 
Navier-Stokes equations the norm is only taken in respect to the velocity components while 
pressure is omitted.
%
\sifitem{Nonlinear System Convergence Absolute}{Logical}
This keyword may be used to enforce absolute convergence measures rather than relative. 
The default is \texttt{False}.
%
\sifitem{Nonlinear System \Idx{Convergence Tolerance}}{Real} 
This keyword gives a criterion to
terminate the nonlinear iteration after the relative change of the norm of the field variable
between two consecutive iterations is small enough $\delta < \epsilon$,
where $\epsilon$ is the value given with this keyword.
%
\sifitem{Nonlinear System Max Iterations}{Integer}
The maxmimum number of nonlinear iterations the
solver is allowed to do.
%
\sifitem{Nonlinear System Newton After Iterations}{Integer} 
Change the nonlinear solver type to
Newton iteration after a number of Picard iterations have been performed. If a given
convergence tolerance between two iterations is met before the iteration count is met,
it will switch the iteration type instead. This applies only to some few solvers (as the Navier-Stokes)
where different linearization strategies are available.
%
\sifitem{Nonlinear System Newton After Tolerance}{Real} 
Change the nonlinear solver type to
Newton iteration, if the relative change of the norm of the field variable meets a
tolerance criterion:
$$
 \delta < \epsilon,
$$
where $\epsilon$ is the value given with this keyword.
%
\sifitem{Nonlinear System \Idx{Relaxation Factor}}{Real} 
Giving this keyword triggers the use
of relaxation in the nonlinear equation solver.
Using a factor below unity is sometimes required to achive convergence of the nonlinear system.
Typical values range between 0.3 and unity. If one must use smaller values for the relaxation
factor some other methods to boost up the convergence might be needed to improve the convergence.
A factor above unity might rarely speed up the convergence. Relaxed variable is defined as follows:
$$
 u^{'}_i = \lambda u_i + (1-\lambda) u_{i-1},
$$
where $\lambda$ is the factor given with this keyword. The default value for the relaxation factor
is unity. 
%
\sifend


Many of the keywords used to control the \texttt{Nonlinear System} have a corresponding 
keyword for the {Steady State}. Basically the operation is similar except the 
reference value for the current solution $u_i$ is the last converged value of the nonlinear system
before starting a new loosely coupled iteration cycle. Otherwise the explanations given above are valid.
\sifbegin
\sifitemnt{Steady State Convergence Measure}{String} 
\sifitemnt{Steady State Norm Degree}{Integer}
\sifitemnt{Steady State Norm Dofs}{Integer}
\sifitemnt{Steady State Convergence Tolerance}{Real}
\sifitemnt{Steady State Relaxation Factor}{Real}
\sifend 
Additionally these keywords are located in the \texttt{Simulation} section
of the command file.
\sifbegin
\sifitem{Steady State Max Iterations}{Integer}
The maximum number of coupled system iterations. For steady state analysis this means
it litelarly, for transient analysis this is the maximum number of iterations within each timestep.
%
\sifitem{Steady State Min Iterations}{Integer}
Sometimes the coupling is such that nontrivial solutions are obtained only after some basic
cycle is repeated. Therefore the user may sometimes need to set also the minimum number of
iterations. Sometimes the steady state loop is also used in a dirty way to do some 
systematic procedures -- for example computing the capacitance matrix, or lumped elastic springs.
Then this value may be set to an a priori known constant value.
\sifend




\bibliography{elmerbib}
\bibliographystyle{plain}

