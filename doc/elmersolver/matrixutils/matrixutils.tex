\chapter{Matrix manipulation utilities}
\noindent


\section{Dirichlet conditions}

In finite element method there are two kinds of boundary conditions. 
The natural boundary condition that may be set by only affecting the r.h.s. of the 
equation and the essential boundary conditions where also the matrix needs to be tampered.
The latter ones are also called Dirichlet boundary conditions. The natural boundary conditions are often
more problem specific so the user is directed to the Models Manual for more details on them.

Technically the Dirichlet conditions in ElmerSolver are set through manipulating only the values
in the matrix rather than its structure. To be more specific, in setting the degree of freedom with index
$i$ the $i$:th row of the matrix is set zero, except for the diagonal which is set to be unity. When also
the r.h.s. of the equation is set to the desired value, the solution will satisfy the Dirichlet condition.

There is a handicap with this procedure which is that the symmetry of the original matrix will be lost.
This may affect the performance of linear system solvers. To ensure to symmetricity of the matrix equation there
are two remedies. Also the column may be zeroed and the known values may be subtracted from the r.h.s. This is envoked
by the keyword 
\sifbegin
  \sifitemnt{Linear System Symmetric}{Logical True}
\sifend
The second option is to eliminate all the rows and columns related to the 
known values. This reduces the size of the matrix but of has an additional cost
as a secondary matrix is created and the values are copied into it. This
feature is activated by the \texttt{EliminateDirichlet} subroutine called before the 
linear system solver with the syntax
\ttbegin
  Before Linsolve = "EliminateDirichlet" "EliminateDirichlet"
\ttend
Both these keywords should locate in the \texttt{Solver} section.


The typical way to give boundary conditions is to define in the \texttt{Boundary condition}
section the values related to the variable under solution, for example. 
\ttbegin
Boundary Condition 1
  Target Boundaries(2) = 2 3
  Temperature = 273
  Velocity 1 = 1.0
End
\ttend
If the variables are not listed in the keyword listing the user shoul also define the type which 
is \texttt{Real}.

There is also a possibility to define boundary conditions for nodes that are created on the fly.
For example, in the case below the two nodes are given a temperature value. 
\ttbegin
Boundary Condition 2
  Target Nodes(2) = 100 200
  Temperature = 293
End
\ttend
A variation of this technique is to give coordinate values which are transformed into 
nodal indexes corresponding to the nearest nodes at the time of first call. 
\ttbegin
  Boundary Condition 3
  Target Coordinates(3,2) = Real 1.0 1.0 1.0 0.5 1.0 0.0
  Temperature = 300
End
\ttend
In the above the keyword dimension is \texttt{Target Boundaries(n,DIM)}.
One should not mix
the conditions defined by \texttt{Target Boundaries}, \texttt{Target Nodes}, and
\texttt{Target Coordinates} in the same boundary condition definition. 

\subsection*{Conditions for bodies}

Usually the Dirichlet conditions are given at objects which have a lower dimension than the 
leading dimension in the geometry, i.e. for 3D problems values are usually fixed only at 2D faces.
However, it is possible also to set the conditions for the bodies. This may be done in the 
\texttt{Body force} section just in the same way as in the \texttt{Boundary condition} section normally. 
\ttbegin
Body Force 1 
  Temperature = 273
  Velocity 1 = 1.0
End
\ttend
When the body force is assigned to a body the values will be fixed as defined. 


\subsection*{Conditional Dirichlet conditions}

Sometimes the Dirichlet conditions should depend on other variables in a way which 
defined whether or not to set the conditions at all. For example, the temperature at a 
boundary should be defined only if the flow is inside the boundary. For outflow the definition
of the temperature is not physically justified. For this kind of purposes the user may give a condition 
that is a \texttt{Real} valued variable that has the name of the field variable followed 
by a string \texttt{Condition}. If this variable is positive the Dirichlet condition is applied,
otherwise not. For example, 
\ttbegin
Boundary Condition 1
  Target Boundaries(2) = 2 3
  Temperature = 273
  Temperature Condition = Equals Velocity 1 
End
\ttend
would define the temperature only if the \texttt{Velocity 1} would be positive.



\section{Periodic boundary conditions}

The periodic boundary conditions in Elmer qre very flexible. In fact they may even be antiperiodic. 
In the \texttt{Boundary Condition} section the following keywords are used to activate the 
periodic BCs.
\sifbegin
  \sifitem{Periodic BC}{Integer}  
   This refers to the counterpart of the periodic boundary condition. This means that 
   periodic boundaries come in pairs, and for the other boundary you only need to give pointer to.
   \sifitem{Anti Periodic BC}{Integer}  
   The system may be also antiperiodic i.e. the absolute value is the same but the sign is different.
   \sifitem{Periodic BC Translate(3)}{Real}
   The periodic boundary is mapped to the other boundary by three different operations: 
   translation, rotatition and scaling. This generality is not usually needed and therefore 
   default value is used. For the translation vector the default is the vector that is obtained when 
   moving in the normal direction of the first boundary until the target boundary is hit. 
   If this is not desired the user may give another translation vector using this keyword.
   \sifitem{Periodic BC Rotate(3)}{Real}
   By default no rotation is performed prior to the mapping of values. This keyword may be used to give 
   the angles of rotation. 
  \sifitem{Periodic BC Scale(3)}{Real}
  By default there is no scaling performed prior to the mapping of values. This keyword may be used to give 
  a scaling vector if this is desired.
  \sifitem{Periodic BC Temperature}{Logical True}
  The user should define the variables that are to be periodic in nature. 
  This is done by attaching their names into logical expressions following 
  the string \texttt{Periodic BC}.
\sifend


\section{Setting nodal loads}

Similarly to the Dirichlet values one may also set nodal loads i.e. entries for the 
r.h.s. of the matrix equation. Generally there are good reasons to avoid the use of nodal loads
as they are mesh dependent. 
There are, however, some uses also for setting nodal loads. For example, in multiphysical couplings
sometimes it may be a good solution to transfer the forces directly in nodal form as this is 
the most accurate way to compute the forces resulting from the discrete system. 

The nodal loads are given exactly as the Dirichlet conditions except that 
a string \texttt{Load} is attached to the name of the variable, for example
\ttbegin
Boundary Condition 1
  Target Boundaries(2) = 2 3
  Temperature Load = Real 1.0
End
\ttend
This may be used also for the on-the-fly conditions created by the \texttt{Target Nodes} and 
\texttt{Target Coordinates} keywords, and to the conditions in the body force section.
Note that there are no conditional nodal loads as there are conditional Dirichlet conditions. 
For nodal loads this may be achieved in a simpler way. 


\section{Computing nodal loads}

In finite element method it is possible to evaluate the nodal loads after the solution 
is computed. This however, requires that the original matrix $A_0$ that has not been 
eliminated for Dirichlet conditions is saved. Then the the nodal forces may be computed from
\begin{equation}
  f = A_0 x - b .
\end{equation}



%Toimii nyt my�s StressSolve:n osalta my�s transienttina tuo 'Constant Bulk System' olettaen, ett� r-ehdot eiv�t
%m�kk�ile massaa ja vaimennusta. Eli jos vain r-ehdoissa on ep�lineaarisuuksia, eik� ole r-ehto vaimennuksia
%niin voipi laittaa vaikka nuo molemmat:

%Constant Bulk System = Logical True
%Linear System Refactorize = Logical False

%Mesh Update 1 DOFs = Equals Displacment 1
%Displacement Load 1 DOFs = Equals Stress 1



\section{Active and passive elements}

In Elmer it is possible to define certain areas of the modeled geometry
passive during the solution. This feature allows also deactivating and
reactivating of the elements. An element being passive means that its
contribution is not included into the global matrix equation. One
could, for example, model two separate bodies heated with different
heating power, and connect them with a third structure only after
suitable time has elapsed. This all could be modeled within a single
simulation.

The geometry of the whole system is meshed as usual, and the passive
elements are only omitted from the equations. The passive definition
is done solverwise and elementwise. The former means that, eg., the
temperature may be passive and the displacements active at the same
element.

The passive property of elements is defined with a real valued
parameter with the name constructed from the name of the variable
followed by \texttt{Passive} in the \texttt{Body Force} section.
\ttbegin
Body Force 1
  Temperature Passive = Variable Time
    Real
      0.0    1.0
      5.0    1.0
      5.2   -1.0
      8.0   -1.0
    End
End
\ttend

When the parameter obtains a value greater than zero the element is
passive. In the above example, the temperature solver (ie. the heat
equation solver) is passive from time instants 0.0 seconds to 5.1 s,
and active after 5.1 seconds.

Note that it is not possible to control components of vector valued
variables separately. Thus the \texttt{Displacement 1},
\texttt{Displacement 2} and \texttt{Displacement 3} variables of
Stress analysis solver are all switched of with the keyword
\texttt{Displacement Passive}. Also, the Navier-Stokes solver is
controlled with the keyword \texttt{Flow Solution Passive} which
affects both the pressure and the velocity components.


\bibliography{elmerbib}
\bibliographystyle{plain}

