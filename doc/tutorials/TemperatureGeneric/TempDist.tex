\chapter{Heat equation -- Temperature field of a solid object}

\modinfo{Directory}{TemperatureGenericGUI}
\modinfo{Solvers}{\Idx{HeatSolve}} 
\modinfo{Tools}{\Idx{ElmerGUI}} 
\modinfo{Dimensions}{3D, Steady-state}

\subsection*{Problem description}

This tutorial tried to demonstrate how to solve the heat equation 
for a generic 3D object. The solid object 
(see figure~\ref{fg:object1}) is heated internally by a heat source.
\begin{figure}
\begin{center}
\includegraphics[width=80mm, viewport=100 100 760 520,clip]{domain}
\caption{Generic object being heated}\label{fg:object1}
\end{center}
\end{figure}
At some part of the boundary the temperature is fixed.
Mathemetically the problem is described by the Poisson equation
\begin{equation}
\left \{
\begin{array}{ccccc}
- \kappa \Delta T &= &\rho f & \mathrm{ in } \, \, & \Omega \\
T&=&0 & \mathrm{ on } & \Gamma
\end{array}
\right .
\end{equation}
where $\kappa$ is the heat conductivity, $T$  is the temperature 
and $f$ is the heat source. It is assumed that density 
and heat conductivity are constants. 

To determine the problem we assume that the part of the boundary is fixed at $T_0=293$~K,
the internal heat generation is, $h=0.01$~W/kg, and use the material properties of aluminium.


\subsection*{Solution procedure}

Start \texttt{ElmerGUI} from command line or by clicking the icon in your desktop. Here we describe 
the essential steps in the ElmerGUI by writing out the clicking procedure. Tabulation generally means that the 
selections are done within the window chosen at the higher level. 

The geometry is given in step format in file \texttt{pump\_carter\_sup.stp}
in the \texttt{samples/step} directory of ElmerGUI, 
This file is kindly provided at the AIM@SHAPE Shape Repository by INRIA.
The heat equation is ideally suited for the finite element method and 
the solution may be found even at meshes that for some other problems
would not be feasible. Therefore you may easily experiment solving the same
problem with different meshes. If you lack either OpenCascade or tetgen you might try to solve the problem 
with the \textt{grd} files \texttt{angle3d.grd, angles3d.grd, 
bench.grd}, or \texttt{cooler.grd}.

The CAD geometry defined by the step file is transformed on-the-fly by OpenCascade library into 
a stl file that 
%may be used by both tetgen (tetlib) and netgen (nglib) mesh denerators to perform volume meshing. 
%In this tutorial we use tetgen.
for which tetgen (tetlib) creates tetrahedral volume discretization.

Load the input file:
\ttbegin
File 
  Open -> pump_carter_sup.stp
\ttend
You should obtain your mesh and may check in the number of element in the \texttt{Model summary}. 
With tetgen the default setting generate 33902 nodes and 110010 tetrahderal elements.
In order to affect the mesh density study the command-line options in the tetgen manual.

The stl description of the mesh only provides with one body and one surface. 
Therefore we need to divide the 
existing surface. First choose the surface by clicking on it (it should turn red) and then 
\ttbegin
Mesh 
  Divide Surface
    Divide
\ttend
You may as well use the red icon with arrows pointing to different directions. 
The algorithm separates all boundaries with more than 20 angle separation between surfaces.
We want to set the temperature at the inside of the holes and in that aim again join three boundaries (see figure~\ref{fg:bcs_chosen}).
For that aim choose the three boundaries as shown in the picture by pressing the \texttt{Ctrl}-key down.
\ttbegin
Mesh 
  Unify Surface
\ttend
\begin{figure}
\begin{center}
\includegraphics[width=100mm]{bcs_chosen}
\caption{The computational mesh showing the three joined boundaries}\label{fg:bcs_chosen}
\end{center}
\end{figure}

After we have the mesh we start to go through the Model menu from the top to bottom. 
In the \texttt{Setup} we choose things related to the whole simulation such as file names, 
time stepping, constants etc.
The simulation is carried out in 3-dimensional cartesian
coordinates and in steady-state. 
Only one steady-state iteration is needed as the case is linear. 
\ttbegin
Model
  Setup 
    Simulation Type = Steady state
    Steady state max. iter = 1
\ttend
Choose \texttt{Accept} to close the window.

In the equation section we choose the relevant equations and parameters related to their solution. 
In this case we'll have one set only one equation -- the heat equation.

When defining Equations and Materials it is possible to assign the to bodies immediately, or to use mouse
selection to assign them later. In this case we have just one body and therefore its easier to assign 
the Equation and Material to it directly, whereas the active boundary is chosen graphically.

For the linear system solvers we are happy to use the defaults. One may however, try out different
preconditioners (ILU1,\ldots), for example.
\ttbegin
Model
  Equation
    Add 
      Name = Heat Equation
      Apply to bodies = Body 0
      Heat Equation
        Active = on
      Add   
      OK
\ttend        

The Material section includes all the material parameters.
They are divided to generic parameters which are direct properties of the material
without making any assumptions on the physical model, such as the mass. Other properties assume
a physical law, such heat conductivity.
We choose Aluminium from the Material library which automatically sets for the needed material properties. 
\ttbegin
Model
  Material
    Add 
      Material library
        Aluminium
      Apply to bodies = Body 0 
      Add 
      OK
\ttend

A Body Force represents the right-hand-side of a equation that in this case represents
the heat source. 
\ttbegin
Model
  Body Force
    Add 
      Name = Heating
      Heat Source = 0.01
      Apply to bodies = Body 0
      Add
      OK
\ttend    

No initial conditions are required in steady state case.

In this case we have only one boundary and set it to room temperature.
First we create the boundary condition
\ttbegin
Model
  BoundaryCondition
    Add 
      Heat Equation
        Temperature = 293.0
      Name = RoomTemp
      Add
      OK
\ttend   
Then we set the boundary properties 
\ttbegin
Model 
  Set boundary properties  
\ttend
Choose the defined group of three boundaries by clicking with the mouse
and apply the condition for this boundary.
\ttbegin
Boundary condition
  RoomTemp
\ttend

For the execution 
ElmerSolver needs the mesh files and the command file. We have know basically defined
all the information for ElmerGUI to write the command file. After writing it we may also visually 
inspect the command file.
\ttbegin
Sif 
  Generate
  Edit -> look how your command file came out  
\ttend

Before we can execute the solver we should save the files in a directory. In saving the project all the
necessary files for restarting the case will be saved to the 
destination directory.
\ttbegin
File 
  Save Project
\ttend

After we have successfully saved the files we may start the solver
\ttbegin
Run
  Start solver
\ttend
A convergence view automatically pops up showing relative changes of each iteration.
As the case is linear only one iteration was required for the solution and the second one
just is needed to check the convergence. The norm of the solution
should be around 389.853~K.

Note: if you face problems in the solution phase and need to edit the setting, always remember to regenerate the
sif file and save the project before execution.


\subsection*{Postprocessing}

To view the results we may use the ElmerPost postprocessor or start the the internal VTK widget as is done 
here, 
\ttbegin
Run
  Postprocessor (VTK)
\ttend
The default configuration shows just the object. To color the surface with the temperature choose
\ttbegin
Surfaces 
  Surface: Temperature
  Apply
\ttend
There seems to be some handing nodes generated by tetgen with zero temperature. Thus, fixing the Min to 
293 and setting crossing the \texttt{Keep limits} box gives better color scale. 
You may also turn on opasity in order to see through the object, 10-20\% is a good value.
This way you'll able to see some isosurfaces that you might want to define.
Some examples of the visualizations may be seen in figure~\ref{fg:vtkpost1}.

\begin{figure}
\begin{center}
\includegraphics[width=120mm]{tempdist} \\
\includegraphics[width=120mm]{tempdist2}
\caption{The temperature distribution of the solid object domain as visualized using the VTK-based postprocessor}\label{fg:vtkpost1}
\end{center}
\end{figure}

\hfill
\mbox{}






